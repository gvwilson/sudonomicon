\documentclass[krantzl]{krantz}

% Language rules.
\usepackage[english]{babel}

% Font encoding.
\usepackage[T1]{fontenc}

% Use Helvetica throughout.
\usepackage{fontspec}
\setmainfont{Helvetica}

% Bibliography.
\usepackage[backend=biber,style=alphabetic,sorting=nyt,maxbibnames=99]{biblatex}
\addbibresource{bibliography.bib}

% https://tex.stackexchange.com/questions/8428/use-bibtex-key-as-the-cite-key
\DeclareFieldFormat{labelalpha}{\thefield{entrykey}}
\DeclareFieldFormat{extraalpha}{}

\setlength{\biblabelsep}{\labelsep}% <-- adjust this to your liking, the standard is 2\labelsep
\defbibenvironment{bibliography}
  {\list
     {\printtext[labelalphawidth]{%
        \printfield{prefixnumber}%
        \printfield{labelalpha}%
        \printfield{extraalpha}}}
     {\setlength{\labelsep}{\biblabelsep}%
      \setlength{\leftmargin}{24pt}%\labelsep
      \setlength{\itemsep}{\bibitemsep}%
      \setlength{\parsep}{\bibparsep}}%
      \renewcommand*{\makelabel}[1]{\bf##1\hss}}
  {\endlist}
  {\item}
\bibparsep3pt

% Include the bibliography in the table of contents.
\usepackage[nottoc,numbib]{tocbibind}

% Build an index.
\usepackage{makeidx}
\makeindex

% Show page boundaries.
% \usepackage[showframe]{geometry}
% \usepackage{geometry}
\usepackage{textcomp}
\DeclareTextCommand{\textapprox}{T1}{\raisebox{0.5ex}{\texttildelow}}

% Some special symbols.
\usepackage{amssymb}
\usepackage{textgreek}

% Format code listings.
% https://tex.stackexchange.com/questions/263032/why-is-listings-frame-width-a-little-larger-then-textwidth
\usepackage{listings}
\lstset{
  basicstyle=\fontsize{8}{10}\ttfamily,
  upquote=true,
  xleftmargin=3.4pt,
  xrightmargin=3.4pt,
  literate={✓}{{\checkmark}}1 {ô}{{\^o}}1 {ü}{{\"u}}1 {…}{{\ldots}}1
}

% Mark keystrokes.
\usepackage{keystroke}

% Include images.
\usepackage{graphicx}

% Make description items cross-referenceable
\def\namedlabel#1#2{\begingroup
    #2%
    \def\@currentlabel{#2}%
    \phantomsection\label{#1}\endgroup
}
\makeatother

% Use numbers for nested lists all the way down and adjust indent.
\usepackage{enumitem}
\setlist[enumerate,1]{label=\arabic*., ref=\arabic*, leftmargin=*}
\setlist[enumerate,2]{label=\arabic*., ref=\arabic*, leftmargin=*}
\setlist[enumerate,3]{label=\arabic*., ref=\arabic*, leftmargin=*}

% Adjust indent for bullet lists.
\setlist[itemize,1]{label=\textbullet, leftmargin=*}
\setlist[itemize,2]{label=\textbullet, leftmargin=*}
\setlist[itemize,3]{label=\textbullet, leftmargin=*}

% Allow multi-page tables.
\usepackage{longtable}

% Center captions.
\usepackage[center]{caption}

% Mark table headers.
\newcommand{\tablehead}[1]{\underline{#1}}

% Figures.
\newcommand{\figimg}[4]{\begin{figure}%
\centering%
\includegraphics[width=\textwidth]{#2}%
\caption{#3}%
\label{#1}%
\end{figure}}

\newcommand{\figpdf}[4]{\begin{figure}%
\centering%
\includegraphics[scale={#4}]{#2}%
\caption{#3}%
\label{#1}%
\end{figure}}

\newcommand{\figpdfhere}[4]{\begin{figure}[h]%
\centering%
\includegraphics[scale={#4}]{#2}%
\caption{#3}%
\label{#1}%
\end{figure}}

% Cross-references.
\newcommand{\appref}[1]{Appendix~\ref{#1}}
\newcommand{\chapref}[1]{Chapter~\ref{#1}}
\newcommand{\figref}[1]{Figure~\ref{#1}}
\newcommand{\secref}[1]{Section~\ref{#1}}
\newcommand{\tblref}[1]{Table~\ref{#1}}

% Glossary items and references.
\newcommand{\glossref}[1]{\textbf{#1}}
\newcommand{\glosskey}[1]{\textbf{#1}}

% Asides
\usepackage[framemethod=default]{mdframed}
\usepackage{footnote}

% Temp box
\usepackage{xcolor}
\newmdenv[skipabove=7pt,
skipbelow=7pt,
rightline=true,
leftline=true,
topline=true,
bottomline=true,
innerleftmargin=-5pt,
innerrightmargin=-5pt,
innertopmargin=-5pt,
leftmargin=0cm,
rightmargin=0cm,
linewidth=.5pt,
innerbottommargin=0pt,
backgroundcolor=black!5]{tBox}

\newtoggle{inbox}
\togglefalse{inbox}

\pretocmd{\footnote}{\iftoggle{inbox}{\stepcounter{footnote}}{\relax}}{}{}

\newcommand{\callouttitle}[1]{\begin{center}{#1}\end{center}\vspace{\baselineskip}}
\newenvironment{callout}{\savenotes\begin{tBox}\begin{quotation}\toggletrue{inbox}\renewcommand{\thempfootnote}{\arabic{footnote}}}{\end{quotation}\vspace{\baselineskip}\end{tBox}\togglefalse{inbox}\spewnotes}
\newenvironment{hint}{\begin{mdframed}\begin{quotation}}{\end{quotation}\end{mdframed}}
\newlength{\tempindent}
\newenvironment{unindented}{%
  \setlength{\tempindent}{\parindent}%
  \setlength{\parindent}{0pt}%
}{%
  \setlength{\parindent}{\tempindent}%
}

% Chapter info.
\newenvironment{chapterinfo}%
{\begin{mdframed}%
\begingroup%
\hyphenpenalty 10000%
\exhyphenpenalty 10000%
}%
{%
\endgroup%
\end{mdframed}%
}

% Unicode characters.
\usepackage{newunicodechar}
\newunicodechar{√}{$\sqrt{}$}
\newunicodechar{≈}{$\approx{}$}
\newunicodechar{⌈}{$\lceil{}$}
\newunicodechar{⌋}{$\rfloor{}$}

% Don't indent footnotes.
\usepackage[hang,flushmargin,bottom]{footmisc}

% URLs as footnotes.
% Always load 'hyperref' last (see link below for explanation).
% https://tex.stackexchange.com/questions/16268/warning-with-footnotes-namehfootnote-xx-has-been-referenced-but-does-not-exi
\newcommand{\hreffoot}[2]{{#1}\footnote{\href{#2}{#2}}}
\usepackage[hidelinks]{hyperref}

% Adapted from Nemilov.cls.
\def\dedication#1{\thispagestyle{empty}\par\vspace*{9pc}\hfil{\large \textbf{\emph{Dedication}}}\hfil\par\vspace*{9pt}%
\hfil {\vrule height.5pt width 7pc}\hfil\par\vspace*{16pt}%
\vbox{\centering {#1}}}

\begin{document}
\title{The Sudonomicon}
\author{Greg Wilson}
\maketitle

\tableofcontents

\mainmatter
\chapter{Introduction}\label{intro}



\section{What This Is}
\begin{itemize}
\item Notes and working examples that instructors can use to perform a lesson\begin{itemize}
\item Do \emph{not} expect novices with no prior Unix experience to be able to learn from them on their own

\end{itemize}


\item Musical analogy\begin{itemize}
\item This is the chord changes and melody

\item We expect instructors to create an arrangement and/or improvise while delivering

\end{itemize}


\item Please see \hreffoot{the license}{./license/} for terms of use,
    the \hreffoot{Code of Conduct}{./conduct/} for community standards,
    and \hreffoot{these guidelines}{./contributing/} for notes on contributing

\end{itemize}
\section{Scope}
\begin{itemize}
\item \hreffoot{Intended audience}{http://teachtogether.tech/en/index.html#s:process-personas}\begin{itemize}
\item Ning did a bachelor’s degree in economics
    and now works as a data analyst for the Ministry of Health

\item They are comfortable working with Unix command-line tools,
    writing data analysis programs in Python,
and downloading data from the web manually

\item Ning wants to understand what happens when they install a package
or run a pipeline in the cloud

\item Their work schedule is unpredictable and highly variable,
    so they need to be able to learn a bit at a time

\end{itemize}


\item Prerequisites\begin{itemize}
\item Unix shell commands covered in \hreffoot{this Software Carpentry lesson}{https://swcarpentry.github.io/shell-novice/}:\begin{itemize}
\item \texttt{pwd}; \texttt{ls}; \texttt{cd}; \texttt{.} and \texttt{..}; \texttt{rm} and \texttt{rmdir}; \texttt{mkdir}; \texttt{touch};
    \texttt{mv}; \texttt{cp}; \texttt{tree}; \texttt{cat}; \texttt{wc}; \texttt{head}; \texttt{tail}; \texttt{less}; \texttt{cut}; \texttt{echo};
    \texttt{history}; \texttt{find}; \texttt{grep}; \texttt{zip}; \texttt{man}

\item current working directory; absolute and relative paths; naming files;
    editing with \texttt{nano}

\item standard input; standard output; standard error; redirection; pipes

\item \texttt{*} and \texttt{?} wildcards; shell variable with \texttt{\$} expansion; \texttt{for} loop

\end{itemize}


\item Python for command-line scripting\begin{itemize}
\item variables; numbers and strings; lists; dictionaries; \texttt{for} and \texttt{while} loops;
\texttt{if}/\texttt{else}; \texttt{with}; defining and calling functions; \texttt{sys.argv}, \texttt{sys.stdin},
and \texttt{sys.stdout}; simple regular expressions; reading JSON data;
reading CSV files using \hreffoot{Pandas}{https://pandas.pydata.org/} or \hreffoot{Polars}{https://pola.rs/}

\item \texttt{pip install}

\end{itemize}


\item \texttt{python -m venv} or \texttt{conda create}

\end{itemize}


\item Learning outcomes\begin{enumerate}
\item Explain the difference between shell variables and environment variables
    and write shell scripts that use each.

\item Create a virtual environment and explain what this actually does.

\item Create \texttt{requirements.txt} file for \hreffoot{\texttt{pip}}{https://pip.pypa.io/en/stable/} and explain version pinning.

\item Explain what a filesystem is (disk partitions, inodes, symbolic links)
    and use \texttt{df}, \texttt{ln}, similar commands to explore with them.

\item Explain what a process is and use commands like \texttt{ps} and \texttt{kill} to explore and manage them.

\item Explain what a job is and use commands like \texttt{jobs}, \texttt{bg}, and \texttt{fg} to manage them.

\item Explain what \texttt{cron} jobs are and how to create them.

\item Explain the difference between a container and a virtual machine.

\item Create and manage Docker images.

\item Explain what ports are and write Python code that uses sockets and HTTP.

\item Explain what certificates are and how they are used to support HTTPS.

\item Explain what key pairs are and how they are stored, and create and manage key pairs.

\item Explain what IP addresses are and how they are resolved.

\item Explain how traditional password authentication works and describe its weaknesses.

\end{enumerate}


\end{itemize}
\section{Setup}
\begin{itemize}
\item Download \hreffoot{the latest release}{https://github.com/gvwilson/sys-tutorial/raw/main/sys-tutorial.zip}

\item Unzip the file in a temporary directory to create:\begin{itemize}
\item \texttt{./site/*.*}: files and directories used in examples

\item \texttt{./src/*.*}: shell scripts and Python programs

\item \texttt{./out/*.*}: expected output for examples

\end{itemize}


\end{itemize}
\section{Acknowledgments}

My thanks to everyone who helped make this tutorial possible:

\begin{itemize}\item Stefan Arentz

\item Julia Evans

\item Robert Kern

\item Matt Panaro

\item Jean-Marc Saffroy
\end{itemize}
\chapter{Processes}\label{process}




Terms defined: 
\glossref{background a process}, \glossref{buffer (verb)}, \glossref{callback function}, \glossref{child process}, \glossref{flush}, \glossref{foreground a process}, \glossref{fork (a process)}, \glossref{parent process}, \glossref{process}, \glossref{process ID}, \glossref{process tree}, \glossref{resume (a process)}, \glossref{signal}, \glossref{signal handler}, \glossref{suspend (a process)}


\section{Program vs. Process}
\begin{itemize}
\item A program is a set of instructions for a computer

\item A \glossref{process} is a running instance of a program\begin{itemize}
\item Code plus variables in memory plus open files plus some IDs

\end{itemize}


\item Tools to manage them were invented when most users only had a single terminal\begin{itemize}
\item But are still useful for working with remote/cloud machines

\end{itemize}


\end{itemize}
\section{Viewing Processes}
\begin{itemize}
\item Use \texttt{ps -a -l} to see currently running processes in terminal\begin{itemize}
\item \texttt{UID}: numeric ID of the user that the process belongs to

\item \texttt{PID}: process’s unique ID

\item \texttt{PPID}: ID of the process’s parent (i.e., the process that created it)

\item \texttt{CMD}: the command the process is running

\end{itemize}


\end{itemize}
\begin{lstlisting}[frame=tblr]
ps -a -l
\end{lstlisting}

\begin{lstlisting}[frame=tblr,backgroundcolor=\color{black!5}]
UID   PID  PPID        F CPU PRI NI        SZ    RSS       TTY       TIME CMD
  0 13215 83470     4106   0  31  0 408655632   9504   ttys001    0:00.02 login -pfl gvwilson /
501 13216 13215     4006   0  31  0 408795632   5424   ttys001    0:00.04 -bash
501 13569 13216     4046   0  31  0 408895008  20864   ttys001    0:00.10 python -m http.server
  0 13577 13216     4106   0  31  0 408766128   1888   ttys001    0:00.01 ps -a -l
\end{lstlisting}

\begin{itemize}
\item Use \texttt{ps -a -x} to see (almost) all processes running on computer\begin{itemize}
\item \texttt{ps -a -x | wc} tells me there are 655 processes running on my laptop right now

\end{itemize}


\end{itemize}
\section{Exercise}
\begin{enumerate}
\item 

What does the \texttt{top} command do?
    What does \texttt{top -o cpu} do?



\item 

What does the \texttt{pgrep} command do?



\end{enumerate}
\section{Parent and Child Processes}
\begin{itemize}
\item Every process is created by another process\begin{itemize}
\item Except the first, which is started automatically when the operating system boots up

\end{itemize}


\item Refer to \glossref{child process} and \glossref{parent process}

\item \texttt{echo \$\$} shows \glossref{process ID} of current process\begin{itemize}
\item \texttt{\$\$} shortcut for current process’s ID because it’s used so often

\end{itemize}


\item \texttt{echo \$PPID} (parent process ID) to get parent

\item \texttt{pstree \$\$} to see \glossref{process tree}

\end{itemize}
\section{Signals}
\begin{itemize}
\item Can send a \glossref{signal} to a process\begin{itemize}
\item “Something extraordinary happened, please deal with it immediately”

\end{itemize}


\item \tblref{process_signals} shows what happened

\end{itemize}
\begin{table}
\centering
\begin{tabular}{llll}
\textbf{\underline{Number}} & \textbf{\underline{Name}} & \textbf{\underline{Default Action}} & \textbf{\underline{Description}} \\
1 & \texttt{SIGHUP} & terminate process & terminal line hangup \\
2 & \texttt{SIGINT} & terminate process & interrupt program \\
3 & \texttt{SIGQUIT} & create core image & quit program \\
4 & \texttt{SIGILL} & create core image & illegal instruction \\
8 & \texttt{SIGFPE} & create core image & floating-point exception \\
9 & \texttt{SIGKILL} & terminate process & kill program \\
11 & \texttt{SIGSEGV} & create core image & segmentation violation \\
12 & \texttt{SIGSYS} & create core image & non-existent system call invoked \\
14 & \texttt{SIGALRM} & terminate process & real-time timer expired \\
15 & \texttt{SIGTERM} & terminate process & software termination signal \\
17 & \texttt{SIGSTOP} & stop process & stop (cannot be caught or ignored) \\
24 & \texttt{SIGXCPU} & terminate process & CPU time limit exceeded \\
25 & \texttt{SIGXFSZ} & terminate process & file size limit exceeded \\
\end{tabular}
\caption{Signals}
\label{process_signals}
\end{table}

\begin{itemize}
\item Create a \glossref{callback function}
    to act as a \glossref{signal handler}

\end{itemize}
\begin{lstlisting}[frame=tblr]
import signal
import sys

COUNT = 0

def handler(sig, frame):
    global COUNT
    COUNT += 1
    print(f"interrupt {COUNT}")
    if COUNT >= 3:
        sys.exit(0)

signal.signal(signal.SIGINT, handler)
print("use Ctrl-C three times")
while True:
    signal.pause()
\end{lstlisting}

\begin{lstlisting}[frame=tblr,backgroundcolor=\color{black!5}]
python src/catch_interrupt.py
use Ctrl-C three times
^Cinterrupt 1
^Cinterrupt 2
^Cinterrupt 3
\end{lstlisting}

\begin{itemize}
\item \texttt{{\textasciicircum}C} shows where user typed Ctrl-C

\end{itemize}
\section{Background Processes}
\begin{itemize}
\item Can run a process in the \glossref{background}\begin{itemize}
\item Only difference is that it isn’t connected to the keyboard (stdin)

\item Can still print to the screen (stdout and stderr)

\end{itemize}


\end{itemize}
\begin{lstlisting}[frame=tblr]
import time

for i in range(3):
    print(f"loop {i}")
    time.sleep(1)
print("loop finished")
\end{lstlisting}

\begin{lstlisting}[frame=tblr]
python src/show_timer.py &
ls site
\end{lstlisting}

\begin{lstlisting}[frame=tblr,backgroundcolor=\color{black!5}]
$ src/show_timer.sh
birds.csv       cert_authority.srl  sandbox         server.pem      species.csv
cert_authority.key  motto.json      server.csr      server_first_cert.pem   yukon.db
cert_authority.pem  motto.txt       server.key      server_first_key.pem
loop 0
$ loop 1
loop 2
loop finished
\end{lstlisting}

\begin{itemize}
\item \texttt{\&} at end of command means “run in the background”

\item So \texttt{ls} command executes immediately

\item But \texttt{show\_timer.py} keeps running until it finishes\begin{itemize}
\item Or needs keyboard input

\end{itemize}


\item Can also start process and then \glossref{suspend} it with Ctrl-Z\begin{itemize}
\item Sends \texttt{SIGSTOP} instead of \texttt{SIGINT}

\end{itemize}


\item Use \texttt{jobs} to see all suspended processes

\item Then \texttt{bg \%\emph{num}} to resume in the background

\item Or \texttt{fg \%\emph{num}} to \glossref{foreground} the process
    to \glossref{resume} its execution

\end{itemize}
\begin{lstlisting}[frame=tblr,backgroundcolor=\color{black!5}]
$ python src/show_timer.py
loop 0
^Z
[1]+  Stopped                 python src/show_timer.py
$ jobs
[1]+  Stopped                 python src/show_timer.py
$ bg
[1]+ python src/show_timer.py &
loop 1
$ loop 2
loop finished
[1]+  Done                    python src/show_timer.py
\end{lstlisting}

\begin{itemize}
\item Note that input and output are mixed together

\end{itemize}
\section{Killing Processes}
\begin{itemize}
\item Use \texttt{kill} to send a signal to a process

\end{itemize}
\begin{lstlisting}[frame=tblr,backgroundcolor=\color{black!5}]
$ python src/show_timer.py
loop 0
^Z
[1]+  Stopped                 python src/show_timer.py
$ kill %1
[1]+  Terminated: 15          python src/show_timer.py
\end{lstlisting}

\begin{itemize}
\item By default, \texttt{kill} sends \texttt{SIGTERM} (terminate process)

\item Variations:\begin{itemize}
\item Give a process ID: \texttt{kill 1234}

\item Send a different signal: \texttt{kill -s INT \%1}

\end{itemize}


\end{itemize}
\begin{lstlisting}[frame=tblr,backgroundcolor=\color{black!5}]
$ python src/show_timer.py
loop 0
^Z
[1]+  Stopped                 python src/show_timer.py
$ kill -s INT %1
[1]+  Stopped                 python src/show_timer.py
$ fg
python src/show_timer.py
Traceback (most recent call last):
  File "/tut/sys/src/show_timer.py", line 5, in <module>
    time.sleep(1)
KeyboardInterrupt
\end{lstlisting}

\section{Fork}
\begin{itemize}
\item \glossref{Fork} creates a duplicate of a process\begin{itemize}
\item Creator is parent, gets process ID of child as return value

\item Child gets 0 as return value (but has something else as its process ID)

\end{itemize}


\end{itemize}
\begin{lstlisting}[frame=tblr]
import os

print(f"starting {os.getpid()}")
pid = os.fork()
if pid == 0:
    print(f"child got {pid} is {os.getpid()}")
else:
    print(f"parent got {pid} is {os.getpid()}")
\end{lstlisting}

\begin{lstlisting}[frame=tblr,backgroundcolor=\color{black!5}]
starting 41618
parent got 41619 is 41618
child got 0 is 41619
\end{lstlisting}

\section{Unpredictability}
\begin{itemize}
\item Output shown above comes from running the program interactively

\item When run as \texttt{python fork.py > temp.out}, the “starting” line may be duplicated\begin{itemize}
\item Programs don’t write directly to the screen

\item Instead, they send text to the operating system for display

\item The operating system \glossref{buffers} output (and input)

\item So the “starting” message may be sitting in a buffer when \texttt{fork} happens

\item In which case both parent and child send it to the operating system to print

\end{itemize}


\item OS decides how much to buffer and when to actually display it

\item Its decision can be affected by what else it is doing

\item So running the same program several times can produce different outputs\begin{itemize}
\item Because your program is only part of a larger sequence of operations

\end{itemize}


\item Dealing with issues like these is
    part of what distinguishes systems programming from “regular” programming

\end{itemize}
\section{Flushing I/O}
\begin{itemize}
\item Can force OS to do I/O \emph{right now} by \glossref{flushing} its buffers

\end{itemize}
\begin{lstlisting}[frame=tblr]
import os
import sys

print(f"starting {os.getpid()}")
sys.stdout.flush()
pid = os.fork()
if pid == 0:
    print(f"child got {pid} is {os.getpid()}")
else:
    print(f"parent got {pid} is {os.getpid()}")
\end{lstlisting}

\begin{lstlisting}[frame=tblr,backgroundcolor=\color{black!5}]
starting 41536
parent got 41537 is 41536
child got 0 is 41537
\end{lstlisting}

\section{Exec}
\begin{itemize}
\item The \texttt{exec} family of functions in \texttt{os} execute a new program
    \emph{inside the calling process}\begin{itemize}
\item Replace existing program and start a new one

\item One of the reasons we need to distinguish “process” from “program”

\end{itemize}


\item Use \texttt{fork}/\texttt{exec} to create a new process and then run a program in it

\end{itemize}
\begin{lstlisting}[frame=tblr]
import os
import sys

print(f"starting {os.getpid()}")
sys.stdout.flush()
pid = os.fork()
if pid == 0:
    os.execl("/bin/echo", "echo", f"child echoing {pid} from {os.getpid()}")
else:
    print(f"parent got {pid} is {os.getpid()}")
\end{lstlisting}

\begin{lstlisting}[frame=tblr,backgroundcolor=\color{black!5}]
starting 46713
parent got 46714 is 46713
child echoing 0 from 46714
\end{lstlisting}

\section{Exercise}
\begin{enumerate}
\item What are the differences between \texttt{os.execl}, \texttt{os.execlp}, and \texttt{os.execv}?
    When and why would you use each?

\end{enumerate}
\chapter{Variables}\label{var}




Terms defined: 
\glossref{environment variable}, \glossref{name collision}, \glossref{operating system}, \glossref{shell}, \glossref{shell variable}, \glossref{source (in shell script)}, \glossref{system call}


\section{Operating System vs. Shell}
\begin{itemize}
\item \glossref{Operating system} (OS) manages your hardware\begin{itemize}
\item Provides a set of \glossref{system calls}
    to make different machines look the same to user-level programs

\end{itemize}


\item \glossref{Command shell} (or just “shell”) is a text UI for interacting with the operating system\begin{itemize}
\item And with user-level programs

\end{itemize}


\item There are many different shells for Unix and Windows\begin{itemize}
\item \hreffoot{Bash}{https://www.gnu.org/software/bash/} and \hreffoot{Zsh}{https://www.zsh.org/} are compatible with the POSIX standard

\item \hreffoot{Fish}{https://fishshell.com/} is nicer, but is not

\item \hreffoot{Nushell}{https://www.nushell.sh/} is even stranger

\item \hreffoot{PowerShell}{https://microsoft.com/powershell} on Windows (and Unix) has a lot of nice features too

\end{itemize}


\item We use Bash in this tutorial but will discuss Nushell later

\end{itemize}
\section{Shell Variables}
\begin{itemize}
\item The shell is a program and programs have variables

\item Create or change with \texttt{\emph{name}=\emph{value}}

\item \glossref{Shell variable} stays in the process that created it\begin{itemize}
\item E.g., that particular running copy of the shell

\end{itemize}


\item Use \texttt{\$\emph{name}} to get value\begin{itemize}
\item \texttt{\$} prefix because people type file and directory names more often

\end{itemize}


\end{itemize}
\begin{lstlisting}[frame=tblr]
# shell_var_outer.sh
WINDOW="neighbor"
echo "outer: window is ${WINDOW}"
bash src/shell_var_inner.sh
\end{lstlisting}

\begin{lstlisting}[frame=tblr]
# shell_var_inner.sh
echo "inner: window is ${window}"
\end{lstlisting}

\begin{lstlisting}[frame=tblr,backgroundcolor=\color{black!5}]
outer: window is neighbor
inner: window is
\end{lstlisting}

\begin{itemize}
\item Note: variables usually written in upper case to distinguish them from filenames\begin{itemize}
\item So use underscores as separators

\end{itemize}


\end{itemize}
\section{Exercise}

What happens if you modify the scripts shown above
to use single quotes instead of double quotes?

\section{Environment Variables}
\begin{itemize}
\item \glossref{Environment variable} is inherited by new processes

\item Use \texttt{export \emph{name}=\emph{value}}

\end{itemize}
\begin{lstlisting}[frame=tblr]
# env_var_outer.sh
WINDOW="neighbor"
export THRESHOLD=0.5
echo "outer: window is ${WINDOW} and threshold is ${THRESHOLD}"
bash src/env_var_inner.sh
\end{lstlisting}

\begin{lstlisting}[frame=tblr]
# env_var_inner.sh
echo "inner: window is ${window} and threshold is ${threshold}"
\end{lstlisting}

\begin{lstlisting}[frame=tblr,backgroundcolor=\color{black!5}]
outer: window is neighbor and threshold is 0.5
inner: window is  and threshold is
\end{lstlisting}

\section{Exercise}

If a child process sets shell or environment variables,
are they visible in the parent once the child finishes executing?

\section{Environment Variables in Programs}
\begin{itemize}
\item Since environment variables are inherited by child processes,
    they are inherited by all programs run from the shell that has them

\end{itemize}
\begin{lstlisting}[frame=tblr]
# env_var_py.sh
WINDOW="neighbor"
export THRESHOLD=0.5
echo "outer: window is ${WINDOW} and threshold is ${THRESHOLD}"
python src/env_var_py.py
\end{lstlisting}

\begin{lstlisting}[frame=tblr]
# env_var_py.py
import os

window = os.getenv("WINDOW", default="not set")
threshold = os.getenv("THRESHOLD", default="not set")
print(f"inner: window is {window} and threshold is {threshold}")
\end{lstlisting}

\begin{lstlisting}[frame=tblr,backgroundcolor=\color{black!5}]
outer: window is neighbor and threshold is 0.5
inner: window is not set and threshold is 0.5
\end{lstlisting}

\section{Inspecting Variables}
\begin{itemize}
\item \texttt{set} on its own lists variables\begin{itemize}
\item And functions, because yes, you can create those in the shell

\item But please don’t: if you need that, write a Python script

\end{itemize}


\item \texttt{env} shows all environment variables

\end{itemize}
\begin{lstlisting}[frame=tblr]
env | cut -d = -f 1 | sort | head -n 10
\end{lstlisting}

\begin{lstlisting}[frame=tblr,backgroundcolor=\color{black!5}]
BASH_SILENCE_DEPRECATION_WARNING
CONDA_DEFAULT_ENV
CONDA_EXE
CONDA_PREFIX
CONDA_PREFIX_1
CONDA_PROMPT_MODIFIER
CONDA_PYTHON_EXE
CONDA_SHLVL
EDITOR
GEM_HOME
\end{lstlisting}

\begin{itemize}
\item Many tools rely on variables to manage configuration\begin{itemize}
\item \hreffoot{NVM}{https://github.com/nvm-sh/nvm} defines 4, \hreffoot{Conda}{https://docs.conda.io/} defines 8

\item No guarantee that their names don’t \glossref{collide}

\end{itemize}


\end{itemize}
\section{Exercise}

The \texttt{os.environ} variable in Python’s \texttt{os} module
is an easy way to get all of the process’s environment variables.
Compare it to what \texttt{env} shows.

\begin{enumerate}
\item 

Are there differences?



\item 

If so, what are they and why do they exist?



\end{enumerate}
\section{Important Environment Variables}
\begin{itemize}
\item 37 environment variables in my current shell

\item Most important are shown in \tblref{var_common}

\end{itemize}
\begin{table}
\centering
\begin{tabular}{lll}
\textbf{\underline{Name}} & \textbf{\underline{Typical Value}} & \textbf{\underline{Purpose}} \\
\texttt{EDITOR} & \texttt{nano} & default text editor \\
\texttt{HOME} & \texttt{/Users/tut} & user’s home directory \\
\texttt{LANG} & \texttt{en\_CA.UTF-8} & user’s preferred (human) language \\
\texttt{PATH} & see below & search path for programs \\
\texttt{PWD} & \texttt{/Users/tut/sys} & present working directory \\
\texttt{SHELL} & \texttt{/bin/bash} & user’s default shell \\
\texttt{TERM} & \texttt{xterm-256color} & type of terminal \\
\texttt{TMPDIR} & \texttt{/var/tmp} & storage for temporary files \\
\texttt{USER} & \texttt{tut} & current user’s name \\
\end{tabular}
\caption{Environment Variables}
\label{var_common}
\end{table}

\section{Search Path}
\begin{itemize}
\item \texttt{PATH} holds a colon-separated list of directories

\item Shell looks in these \emph{in order} to find commands

\item Reading at them all on one line is difficult, so use \texttt{tr} to split

\end{itemize}
\begin{lstlisting}[frame=tblr]
echo $PATH | tr : '\n'
\end{lstlisting}

\begin{lstlisting}[frame=tblr,backgroundcolor=\color{black!5}]
/Users/tut/google-cloud-sdk/bin
/Users/tut/conda/envs/sys/bin
/Users/tut/conda/condabin
/Users/tut/.nvm/versions/node/v20.8.0/bin
/Users/tut/bin
/Applications/Postgres.app/Contents/Versions/14/bin
/Users/tut/go/bin
/usr/local/bin
/System/Cryptexes/App/usr/bin
/usr/bin
/bin
/usr/sbin
/sbin
/var/run/com.apple.security.cryptexd/codex.system/bootstrap/usr/local/bin
/var/run/com.apple.security.cryptexd/codex.system/bootstrap/usr/bin
/var/run/com.apple.security.cryptexd/codex.system/bootstrap/usr/appleinternal/bin
/Library/Apple/usr/bin
/Library/TeX/texbin
/usr/local/bin
\end{lstlisting}

\begin{itemize}
\item Notice \texttt{/Users/tut/bin}

\item Common to have a \texttt{$\sim$/bin} directory with the user’s own utilities

\end{itemize}
\section{Adding to the Search Path}
\begin{itemize}
\item Shell variables (of both kinds) are just strings

\item So add to search path by redefining the variable\begin{itemize}
\item New directory at the from

\item Entire old value at the back

\end{itemize}


\end{itemize}
\begin{lstlisting}[frame=tblr]
export PATH="/tmp/bin:${PATH}"
echo $PATH | tr : '\n' | head -n 5
\end{lstlisting}

\begin{lstlisting}[frame=tblr,backgroundcolor=\color{black!5}]
/tmp/bin
/Users/gregwilson/google-cloud-sdk/bin
/Users/gregwilson/conda/envs/sys/bin
/Users/gregwilson/conda/condabin
/Users/gregwilson/.gem/ruby/3.1.2/bin
\end{lstlisting}

\section{Exercise}

Removing a directory from \texttt{PATH} is harder than adding one.
Write a shell script that:

\begin{enumerate}
\item Splits \texttt{PATH} on colons to put one entry on each line.

\item Uses \texttt{grep} to remove the undesired line.

\item Uses \texttt{paste -s -d :} to recombine the lines.

\item Uses command interpolation to assign the result back to \texttt{PATH}.

\end{enumerate}

This exercise may remind you why
complicated operations should be done in Python rather than in the shell.

\section{Startup Files}
\begin{itemize}
\item Bash shell runs commands in \texttt{$\sim$/.bash\_profile} for login shells

\item Bash shell runs commands in \texttt{$\sim$/.bashrc} for interactive shells

\item Yes, the terminology is confusing

\item Common to have \texttt{$\sim$/.bash\_profile} \glossref{source} \texttt{$\sim$/.bashrc}\begin{itemize}
\item I.e., run those commands in the current shell

\end{itemize}


\end{itemize}
\begin{lstlisting}[frame=tblr]
source $HOME/.bashrc
\end{lstlisting}

\section{Command Interpolation}
\begin{itemize}
\item Can use \texttt{outer \$(\emph{inner})} to run \texttt{inner} and use its output as arguments to \texttt{outer}

\item Long-winded way to count lines in some text files

\end{itemize}
\begin{lstlisting}[frame=tblr]
wc -l $(ls src/*.text)
\end{lstlisting}

\begin{lstlisting}[frame=tblr,backgroundcolor=\color{black!5}]
12 src/ctrl_z_background.text
12 src/kill_int.text
 6 src/kill_process.text
30 total
\end{lstlisting}

\chapter{HTTP}\label{http}




Terms defined: 
\glossref{character encoding}, \glossref{concurrency}, \glossref{HTTP}, \glossref{header (of HTTP request or response)}, \glossref{HTTP method}, \glossref{HTTP request}, \glossref{HTTP response}, \glossref{HTTP status code}, \glossref{JavaScript Object Notation}, \glossref{local server}, \glossref{localhost}, \glossref{MIME type}, \glossref{port}, \glossref{query parameter}, \glossref{refactor}, \glossref{resolve (a path)}, \glossref{sandbox}, \glossref{static file}, \glossref{UTF-8}, \glossref{web scraping}


\section{Start with Something Simple}
\begin{lstlisting}[frame=tblr]
import requests

url = "https://gvwilson.github.io/safety-tutorial/site/motto.txt"
response = requests.get(url)
print(f"status code: {response.status_code}")
print(f"body:\n{response.text}")
\end{lstlisting}

\begin{lstlisting}[frame=tblr,backgroundcolor=\color{black!5}]
status code: 200
body:
Start where you are, use what you have, help who you can.
\end{lstlisting}

\begin{itemize}
\item Use the \hreffoot{\texttt{requests}}{https://requests.readthedocs.io/en/latest/} module to send an \glossref{HTTP} \glossref{request}

\item The URL identifies the file we want\begin{itemize}
\item Though as we’ll see, the server can interpret it differently

\end{itemize}


\item Response includes:\begin{itemize}
\item \glossref{HTTP status code} such as 200 (OK) or 404 (Not Found)

\item The text of the response

\end{itemize}


\end{itemize}
\section{What Just Happened}
\begin{itemize}
\item \figref{http_lifecycle} shows what happened

\end{itemize}
\figpdf{http_lifecycle}{http/./http_lifecycle.pdf}{Lifecycle of an HTTP request and response}{0.8}
\begin{itemize}
\item Open a connection to the server

\item Send an \glossref{HTTP request} for the file we want

\item Server creates a \glossref{response} that includes the contents of the file

\item Sends it back

\item \texttt{requests} parses the response and creates a Python object for us

\end{itemize}
\section{Request Structure}
\begin{lstlisting}[frame=tblr]
import requests
from requests_toolbelt.utils import dump

url = "https://gvwilson.github.io/safety-tutorial/site/motto.txt"
response = requests.get(url)
data = dump.dump_all(response)
print(str(data, "utf-8"))
\end{lstlisting}

\begin{lstlisting}[frame=tblr,backgroundcolor=\color{black!5}]
GET /safety-tutorial/site/motto.txt HTTP/1.1
Host: gvwilson.github.io
User-Agent: python-requests/2.31.0
Accept-Encoding: gzip, deflate
Accept: */*
Connection: keep-alive
\end{lstlisting}

\begin{itemize}
\item First line is \glossref{method}, URL, and protocol version

\item Every HTTP request can have \glossref{headers} with extra information\begin{itemize}
\item And optionally data being uploaded

\end{itemize}


\item Yes, it’s all just text\begin{itemize}
\item Except for uploaded data, which is just bytes

\end{itemize}


\end{itemize}
\section{Response Structure}
\begin{lstlisting}[frame=tblr]
import requests

url = "https://gvwilson.github.io/web-tutorial/site/motto.txt"
response = requests.get(url)
for key, value in response.headers.items():
    print(f"{key}: {value}")
\end{lstlisting}

\begin{lstlisting}[frame=tblr,backgroundcolor=\color{black!5}]
Connection: keep-alive
Content-Length: 5142
Server: GitHub.com
Content-Type: text/html; charset=utf-8
permissions-policy: interest-cohort=()
ETag: W/"65c56fc7-239b"
Content-Security-Policy: default-src 'none'; style-src 'unsafe-inline'; img-src data:; connect-src 'self'
Content-Encoding: gzip
X-GitHub-Request-Id: A08C:5A357:7CF794:9CEA06:65D13923
Accept-Ranges: bytes
Date: Sat, 17 Feb 2024 22:54:27 GMT
Via: 1.1 varnish
Age: 0
X-Served-By: cache-yyz4563-YYZ
X-Cache: MISS
X-Cache-Hits: 0
X-Timer: S1708210467.301651,VS0,VE20
Vary: Accept-Encoding
X-Fastly-Request-ID: cb16df2dfa73aaf6de87924c743dd1e50a0ce570
\end{lstlisting}

\begin{itemize}
\item Every HTTP response also has with extra information\begin{itemize}
\item Does \emph{not} include status code: that appears in the first line

\end{itemize}


\item Most important for now are:\begin{itemize}
\item \texttt{Content-Length}: number of bytes in response data (i.e., how much to read)

\item \texttt{Content-Type}: \glossref{MIME type} of data (e.g., \texttt{text/plain})

\end{itemize}


\item From now on we will only show interesting headers

\end{itemize}
\section{Exercise}
\begin{enumerate}
\item 

Add header called \texttt{Studying} with the value \texttt{safety}
    to the \texttt{requests} script shown above.
    Does it make a difference to the response?
    Should it?



\item 

What is the difference between the \texttt{Content-Type} and the \texttt{Content-Encoding} headers?



\end{enumerate}
\section{When Things Go Wrong}
\begin{lstlisting}[frame=tblr]
import requests

url = "https://gvwilson.github.io/web-tutorial/site/nonexistent.txt"
response = requests.get(url)
print(f"status code: {response.status_code}")
print(f"body length: {len(response.text)}")
\end{lstlisting}

\begin{lstlisting}[frame=tblr,backgroundcolor=\color{black!5}]
status code: 404
body length: 9115
\end{lstlisting}

\begin{itemize}
\item The 404 status code tells us something went wrong

\item The 9 kilobyte response is an HTML page with an embedded image (the GitHub logo)

\item The page contains human-readable error messages\begin{itemize}
\item But we have to know page format to pull them out

\end{itemize}


\end{itemize}
\section{Exercise}

Look at \hreffoot{this list of HTTP status codes}{https://en.wikipedia.org/wiki/List_of_HTTP_status_codes}.

\begin{enumerate}
\item 

What is the difference between status code 403 and status code 404?



\item 

What is status code 418 used for?



\item 

Under what circumstances would you expect to get a response whose status code is 505?



\end{enumerate}
\section{Getting JSON}
\begin{lstlisting}[frame=tblr]
import requests

url = "https://gvwilson.github.io/web-tutorial/site/motto.json"
response = requests.get(url)
print(f"status code: {response.status_code}")
print(f"body as text: {len(response.text)} bytes")
as_json = response.json()
print(f"body as JSON:\n{as_json}")
\end{lstlisting}

\begin{lstlisting}[frame=tblr,backgroundcolor=\color{black!5}]
status code: 200
body as text: 107 bytes
body as JSON:
{'first': 'Start where you are', 'second': 'Use what you have', 'third': 'Help who you can'}
\end{lstlisting}

\begin{itemize}
\item Parsing data out of HTML is called \glossref{web scraping}\begin{itemize}
\item Painful and error prone

\end{itemize}


\item Better to have the server return data as data\begin{itemize}
\item Preferred format these days is \glossref{JSON}

\item So common that \texttt{requests} has built-in support

\end{itemize}


\item Unfortunately, there is no standard for representing tabular data as JSON \figref{http_json_tables}\begin{itemize}
\item A list with one list with N column names + N lists of values?

\item A list with N dictionaries, all with the same keys?

\item A dictionary with column names and lists of values, all the same length?

\end{itemize}


\end{itemize}
\figpdf{http_json_tables}{http/./http_json_tables.pdf}{Representing tables as JSON}{0.8}
\section{Exercise}

Write a \texttt{requests} script that gets the current location and crew roster
of \hreffoot{the International Space Station}{http://api.open-notify.org/}.

\section{Local Web Server}
\begin{itemize}
\item Pushing files to GitHub so that we can use them is annoying

\item And we want to show how to make things \emph{wrong} so that we can then make them \emph{right}

\item Use Python’s \hreffoot{\texttt{http.server}}{https://docs.python.org/3/library/http.server.html} module
    to run a \glossref{local server}

\end{itemize}
\begin{lstlisting}[frame=tblr]
python -m http.server -d site
\end{lstlisting}

\begin{itemize}
\item Host name is \glossref{\texttt{localhost}}

\item Uses \glossref{port} 8000 by default\begin{itemize}
\item So URLs look like \texttt{http://localhost:8000/path/to/file}

\end{itemize}


\item \texttt{-d site} tells the server to use \texttt{site} as its root directory

\item Use this local server for the next few examples\begin{itemize}
\item Build our own server later on to show how it works

\end{itemize}


\end{itemize}
\section{Talk to Local Server}
\begin{lstlisting}[frame=tblr]
import requests

URL = "http://localhost:8000/motto.txt"

response = requests.get(URL)
print(f"status code: {response.status_code}")
print(f"body:\n{response.text}")
\end{lstlisting}

\begin{lstlisting}[frame=tblr,backgroundcolor=\color{black!5}]
::ffff:127.0.0.1 - - [18/Feb/2024 09:12:24] "GET /motto.txt HTTP/1.1" 200 -
status code: 200
body:
Start where you are, use what you have, help who you can.
\end{lstlisting}

\begin{itemize}
\item \glossref{Concurrent} systems are hard to debug\begin{itemize}
\item Multiple streams of activity

\item Order may change from run to run

\item Usually easiest to run each process in its own terminal window

\end{itemize}


\end{itemize}
\section{Our Own File Server}
\begin{lstlisting}[frame=tblr]
class RequestHandler(BaseHTTPRequestHandler):
    def do_GET(self):
        try:
            url_path = self.path.lstrip("/")
            full_path = Path.cwd().joinpath(url_path)
            print(f"'{self.path}' => '{full_path}'")
            if not full_path.exists():
                raise ServerException(f"{self.path} not found")
            elif not full_path.is_file():
                raise ServerException(f"{self.path} not file")
            else:
                self.handle_file(self.path, full_path)
        except Exception as msg:
            self.handle_error(msg)
\end{lstlisting}

\begin{itemize}
\item Our \texttt{RequestHandler} handles a single \texttt{GET} request

\item Combine working directory with requested file path to get local path to file

\item Return that if it exists and is a file or raise an error

\end{itemize}
\section{Support Code}
\begin{itemize}
\item Serve files

\end{itemize}
\begin{lstlisting}[frame=tblr]
    def send_content(self, content, status):
        self.send_response(int(status))
        self.send_header("Content-Type", "text/html; charset=utf-8")
        self.send_header("Content-Length", str(len(content)))
        self.end_headers()
        self.wfile.write(content)
\end{lstlisting}

\begin{itemize}
\item Handle errors

\end{itemize}
\begin{lstlisting}[frame=tblr]
ERROR_PAGE = """\
<html>
  <head><title>Error accessing {path}</title></head>
  <body>
    <h1>Error accessing {path}: {msg}</h1>
  </body>
</html>
"""
\end{lstlisting}

\begin{lstlisting}[frame=tblr]
    def handle_error(self, msg):
        content = ERROR_PAGE.format(path=self.path, msg=msg)
        content = bytes(content, "utf-8")
        self.send_content(content, HTTPStatus.NOT_FOUND)
\end{lstlisting}

\begin{itemize}
\item Define our own exceptions so we’re sure we’re only catching what we expect

\end{itemize}
\begin{lstlisting}[frame=tblr]
class ServerException(Exception):
    pass
\end{lstlisting}

\section{Running Our File Server}
\begin{lstlisting}[frame=tblr]
if __name__ == "__main__":
    os.chdir(sys.argv[1])
    serverAddress = ("", 8000)
    server = HTTPServer(serverAddress, RequestHandler)
    print(f"serving in {os.getcwd()}...")
    server.serve_forever()
\end{lstlisting}

\begin{itemize}
\item And then get \texttt{motto.txt} as before

\end{itemize}
\section{Built-in Safety}
\begin{itemize}
\item Modify \texttt{requests} script to take URL as command-line parameter

\end{itemize}
\begin{lstlisting}[frame=tblr]
import requests
import sys

URL = sys.argv[1]

response = requests.get(URL)
print(f"status code: {response.status_code}")
print(f"body:\n{response.text}")
\end{lstlisting}

\begin{itemize}
\item Add a sub-directory to \texttt{site} called \texttt{sandbox} with a file \texttt{example.txt}\begin{itemize}
\item Called a \glossref{sandbox} because it’s a safe place to play

\end{itemize}


\item Serve that sub-directory

\end{itemize}
\begin{lstlisting}[frame=tblr]
python src/file_server_unsafe.py site/sandbox
\end{lstlisting}

\begin{itemize}
\item Can get files from that directory

\end{itemize}
\begin{lstlisting}[frame=tblr]
python src/get_url.py http://localhost:8000/example.txt
\end{lstlisting}

\begin{lstlisting}[frame=tblr,backgroundcolor=\color{black!5}]
'/example.txt' => '/tut/safety/site/sandbox/example.txt'
127.0.0.1 - - [21/Feb/2024 06:04:32] "GET /example.txt HTTP/1.1" 200 -
\end{lstlisting}

\begin{lstlisting}[frame=tblr,backgroundcolor=\color{black!5}]
status code: 200
body:
example file
\end{lstlisting}

\begin{itemize}
\item But not from parent directory (which isn’t part of sandbox)

\end{itemize}
\begin{lstlisting}[frame=tblr]
python src/requests_local_url.py http://localhost:8000/motto.txt
\end{lstlisting}

\begin{lstlisting}[frame=tblr,backgroundcolor=\color{black!5}]
'/motto.txt' => '/tut/safety/site/sandbox/motto.txt'
127.0.0.1 - - [21/Feb/2024 06:04:38] "GET /motto.txt HTTP/1.1" 404 -
\end{lstlisting}

\begin{lstlisting}[frame=tblr,backgroundcolor=\color{black!5}]
status code: 404
body:
<html>
  <head><title>Error accessing /motto.txt</title></head>
  <body>
    <h1>Error accessing /motto.txt: /motto.txt not found</h1>
  </body>
</html>
\end{lstlisting}

\begin{itemize}
\item \texttt{requests} strips the leading \texttt{..} off the path before sending it

\item And if we try that URL in the browser, same thing happens

\item So we’re safe, right?

\end{itemize}
\section{Introducing netcat}
\begin{itemize}
\item \hreffoot{\texttt{netcat}}{https://en.wikipedia.org/wiki/Netcat} (often just \texttt{nc}) is a computer networking tool

\item Open a connection, send exactly what the user types, and show exactly what is sent in response

\end{itemize}
\begin{lstlisting}[frame=tblr]
nc localhost 8000
\end{lstlisting}

\begin{lstlisting}[frame=tblr,backgroundcolor=\color{black!5}]
GET /example.txt HTTP/1.1
\end{lstlisting}

\begin{lstlisting}[frame=tblr,backgroundcolor=\color{black!5}]
HTTP/1.0 200 OK
Server: BaseHTTP/0.6 Python/3.12.1
Date: Thu, 22 Feb 2024 18:37:37 GMT
Content-Type: text/html; charset=utf-8
Content-Length: 13

example file
\end{lstlisting}

\begin{itemize}
\item Let’s see what happens if we \emph{do} send a URL with \texttt{..} in it

\end{itemize}
\begin{lstlisting}[frame=tblr,backgroundcolor=\color{black!5}]
GET ../motto.txt HTTP/1.1
\end{lstlisting}

\begin{lstlisting}[frame=tblr,backgroundcolor=\color{black!5}]
HTTP/1.0 200 OK
Server: BaseHTTP/0.6 Python/3.12.1
Date: Thu, 22 Feb 2024 18:38:50 GMT
Content-Type: text/html; charset=utf-8
Content-Length: 58

Start where you are, use what you have, help who you can.
\end{lstlisting}

\begin{itemize}
\item We shouldn’t be able to see files outside the sandbox

\item But if someone doesn’t strip out the \texttt{..} characters, users can escape

\end{itemize}
\section{Exercise}

The shortcut \texttt{$\sim$\emph{username}} means
“the specified user’s home directory” in the shell,
while \texttt{$\sim$} on its own means “the current user’s home directory”.
Create a file called \texttt{test.txt} in your home directory
and then try to get \texttt{$\sim$/test.txt} using your browser,
\texttt{requests},
and \texttt{netcat}.
What happens with each and why?

\section{A Safer File Server}
\begin{lstlisting}[frame=tblr]
    def handle_file(self, given_path, full_path):
        try:
            resolved_path = str(full_path.resolve())
            sandbox = str(Path.cwd().resolve())
            if not resolved_path.startswith(sandbox):
                raise ServerException(f"Cannot access {given_path}")
            with open(full_path, "rb") as reader:
                content = reader.read()
            self.send_content(content, HTTPStatus.OK)
        except FileNotFoundError:
            raise ServerException(f"Cannot find {given_path}")
        except IOError:
            raise ServerException(f"Cannot read {given_path}")
\end{lstlisting}

\begin{itemize}
\item \glossref{Resolve} the constructed path

\item Then check that it’s below the current working directory (i.e., the sandbox)

\item And fail if it isn’t\begin{itemize}
\item Using our own \texttt{ServerException} guarantees that all errors are handled the same way

\end{itemize}


\end{itemize}
\section{Exercise}

\glossref{Refactor} the \texttt{do\_GET} and \texttt{handle\_file} methods in \texttt{RequestHandler}
so that all checks are in one place.
Does this make the code easier to understand overall?
Do you think making code easier to understand also makes it safer?

\section{Serving Data}
\begin{itemize}
\item Rarely have JSON lying around as \glossref{static files}

\item More common to have either CSV or a database

\end{itemize}
\begin{lstlisting}[frame=tblr]
head -n 10 site/birds.csv
\end{lstlisting}

\begin{lstlisting}[frame=tblr,backgroundcolor=\color{black!5}]
loc_id,latitude,longitude,region,year,month,day,species_id,num
L13476859,60.8606726,-135.2015181,CA-YT,2021,2,1,redcro,3.0
L13476859,60.8606726,-135.2015181,CA-YT,2021,2,1,rebnut,1.0
L13476859,60.8606726,-135.2015181,CA-YT,2021,2,1,comred,13.0
L13476859,60.8606726,-135.2015181,CA-YT,2021,2,1,dowwoo,1.0
L13476859,60.8606726,-135.2015181,CA-YT,2021,2,1,bkcchi,3.0
L13476859,60.8606726,-135.2015181,CA-YT,2021,2,1,haiwoo,1.0
L13476859,60.8606726,-135.2015181,CA-YT,2021,2,8,nobird,
L13476859,60.8606726,-135.2015181,CA-YT,2021,2,15,rebnut,2.0
L13476859,60.8606726,-135.2015181,CA-YT,2021,2,15,bkcchi,3.0
\end{lstlisting}

\begin{itemize}
\item Modify server to generate it dynamically

\item Main program

\end{itemize}
\begin{lstlisting}[frame=tblr]
def main():
    sandbox, filename = sys.argv[1], sys.argv[2]
    os.chdir(sandbox)
    df = pl.read_csv(filename)
    serverAddress = ("", 8000)
    server = BirdServer(df, serverAddress, RequestHandler)
    server.serve_forever()
\end{lstlisting}

\begin{itemize}
\item Create our own server class because we want to pass the dataframe in the constructor

\end{itemize}
\begin{lstlisting}[frame=tblr]
class BirdServer(HTTPServer):
    def __init__(self, data, server_address, request_handler):
        super().__init__(server_address, request_handler)
        self._data = data
\end{lstlisting}

\begin{itemize}
\item \texttt{do\_GET} converts the dataframe to JSON (will modify later to do more than this)

\end{itemize}
\begin{lstlisting}[frame=tblr]
class RequestHandler(BaseHTTPRequestHandler):
    def do_GET(self):
        result = self.server._data.write_json(row_oriented=True)
        self.send_content(result, HTTPStatus.OK)
\end{lstlisting}

\begin{itemize}
\item \texttt{send\_content} \glossref{encodes} the JSON string as \glossref{UTF-8}
    and sets the MIME type to \texttt{application/json}

\end{itemize}
\begin{lstlisting}[frame=tblr]
    def send_content(self, content, status):
        content = bytes(content, "utf-8")
        self.send_response(int(status))
        self.send_header("Content-Type", "application/json; charset=utf-8")
        self.send_header("Content-Length", str(len(content)))
        self.end_headers()
        self.wfile.write(content)
\end{lstlisting}

\begin{itemize}
\item Can view in browser at \texttt{http://localhost:8000} or use \texttt{requests} to fetch as before

\end{itemize}
\section{Slicing Data}
\begin{itemize}
\item URL can contain \glossref{query parameters}

\item Want \texttt{http://localhost:8000/?year=2021\&species=rebnut} to select red-breasted nuthatches in 2021

\item Put slicing in a method of its own

\end{itemize}
\begin{lstlisting}[frame=tblr]
    def do_GET(self):
        result = self.filter_data()
        as_json = result.to_json(orient="records")
        self.send_content(as_json, HTTPStatus.OK)
\end{lstlisting}

\begin{itemize}
\item Use \texttt{urlparse} and \texttt{parse\_qs} from \hreffoot{\texttt{urllib.parse}}{https://docs.python.org/3/library/urllib.parse.html} to get query parameters\begin{itemize}
\item (Key, list) dictionary

\end{itemize}


\item Then filter data as requested

\end{itemize}
\begin{lstlisting}[frame=tblr]
    def filter_data(self):
        params = parse_qs(urlparse(self.path).query)
        result = self.server._data
        if "species" in params:
            species = params["species"][0]
            result = result[result["species_id"] == species]
        if "year" in params:
            year = int(params["year"][0])
            result = result[result["year"] == year]
        return result
\end{lstlisting}

\section{Exercise}
\begin{enumerate}
\item 

Write a function that takes a URL as input
    and returns a dictionary whose keys are the query parameters’ names
    and whose values are lists of their values.
    Do you now see why you should use the library function to do this?



\item 

Modify the server so that clients can specify which columns they want returned
    as a comma-separated list of names.
    If the client asks for a column that doesn’t exist, ignore it.



\item 

Modify your solution to the previous exercise so that
    if the client asks for a column that doesn’t exist
    the server returns a status code 400 (Bad Request)
    and a JSON blog with two keys:
    \texttt{status\_code} (set to 400)
    and \texttt{error\_message} (set to something informative).
    Explain why the server should return JSON rather than HTML in the case of an error.



\end{enumerate}
\chapter{Networks}\label{network}




FIXME: describe network tools

\begin{itemize}
\item See \hreffoot{Issue 20}{https://github.com/gvwilson/sys-tutorial/issues/20}

\end{itemize}
\chapter{Virtualization}\label{virt}




Terms defined: 
\glossref{cache}, \glossref{daemon}, \glossref{Docker container}, \glossref{Docker image}, \glossref{layer (of Docker image)}, \glossref{tag (a Docker image)}, \glossref{Dockerfile}, \glossref{mount}, \glossref{virtual environment}


\section{Virtual Environments}
\begin{itemize}
\item If two directories \texttt{A} and \texttt{B} contain a program \texttt{xyz}
    and \texttt{A} comes before \texttt{B} in the user’s \texttt{PATH},
    the command \texttt{xyz} will run \texttt{A/xyz} instead of \texttt{B/xyz}

\item This is how \glossref{virtual environments} work

\end{itemize}
\begin{lstlisting}[frame=tblr]
echo $PATH | tr : '\n' | grep conda
echo "python is" $(which python)
\end{lstlisting}

\begin{lstlisting}[frame=tblr,backgroundcolor=\color{black!5}]
/Users/tut/conda/envs/sys/bin
/Users/tut/conda/condabin
python is /Users/tut/conda/envs/sys/bin/python
\end{lstlisting}

\begin{itemize}
\item Virtual environment is initially a minimal Python installation

\item Installing new packages puts them in the environment’s directory

\end{itemize}
\section{Package Installation}
\begin{enumerate}
\item Create a new virtual environment called \texttt{example}: \texttt{conda create -n example python=3.12}

\item Activate that virtual environment: \texttt{conda activate example}

\item Install the \texttt{faker} package: \texttt{pip install faker}

\end{enumerate}
\begin{lstlisting}[frame=tblr]
find $HOME/conda/envs/example -name faker
\end{lstlisting}

\begin{lstlisting}[frame=tblr,backgroundcolor=\color{black!5}]
/Users/tut/conda/envs/example/bin/faker
/Users/tut/conda/envs/example/lib/python3.12/site-packages/faker
\end{lstlisting}

\begin{itemize}
\item The script in \texttt{bin} loads the module and runs it

\end{itemize}
\begin{lstlisting}[frame=tblr]
#!/Users/gregwilson/conda/envs/example/bin/python3.12
# -*- coding: utf-8 -*-
import re
import sys
from faker.cli import execute_from_command_line
if __name__ == '__main__':
    sys.argv[0] = re.sub(r'(-script\.pyw|\.exe)?$', '', sys.argv[0])
    sys.exit(execute_from_command_line())
\end{lstlisting}

\begin{itemize}
\item The directory under \texttt{site-packages} has 642 Python files (as of version 24.3.0)

\item The \texttt{python} in the virtual environment’ \texttt{bin} directory
    knows to look in that environment’s \texttt{site-packages} directory

\end{itemize}
\section{Exercises}

What is the \texttt{re.sub} call in the \texttt{faker} script doing and why?

\section{Limits of Virtual Environments}
\begin{itemize}
\item \texttt{conda} (and equivalents like \texttt{python -m venv}) work for Python

\item But what if you need an isolated environment for several languages at once?\begin{itemize}
\item Rust, JavaScript, and other languages all have their own solutions

\end{itemize}


\item And what if you want other people to be able to reproduce that environment?

\end{itemize}
\section{Docker}
\begin{itemize}
\item \hreffoot{Docker}{https://www.docker.com/} solves these problems (and others)

\item Define an \glossref{image} with its own copy of the operating system, filesystem, etc.

\item Run it in a \glossref{container} that is isolated from the rest of your computer

\end{itemize}
\begin{lstlisting}[frame=tblr]
docker image ls
\end{lstlisting}

\begin{lstlisting}[frame=tblr,backgroundcolor=\color{black!5}]
REPOSITORY   TAG       IMAGE ID   CREATED   SIZE
\end{lstlisting}

\begin{lstlisting}[frame=tblr]
docker container ls
\end{lstlisting}

\begin{lstlisting}[frame=tblr,backgroundcolor=\color{black!5}]
CONTAINER ID   IMAGE     COMMAND   CREATED   STATUS    PORTS     NAMES
\end{lstlisting}

\begin{itemize}
\item Because we haven’t created or run anything yet

\end{itemize}
\section{Common Error Message}
\begin{itemize}
\item Docker requires a \glossref{daemon} process
    to be running in the background to start images

\end{itemize}
\begin{lstlisting}[frame=tblr]
docker image ls
\end{lstlisting}

\begin{lstlisting}[frame=tblr,backgroundcolor=\color{black!5}]
Cannot connect to the Docker daemon at unix:///Users/tut/.docker/run/docker.sock.
Is the docker daemon running?
\end{lstlisting}

\section{Running a Container}
\begin{lstlisting}[frame=tblr,backgroundcolor=\color{black!5}]
$ docker container run ubuntu:latest
Unable to find image 'ubuntu:latest' locally
latest: Pulling from library/ubuntu
bccd10f490ab: Pull complete
Digest: sha256:77906da86b60585ce12215807090eb327e7386c8fafb5402369e421f44eff17e
Status: Downloaded newer image for ubuntu:latest

$ docker container ls

$ docker container ls -a
CONTAINER ID   IMAGE           COMMAND       CREATED          STATUS                     PORTS     NAMES
741bb295734f   ubuntu:latest   "/bin/bash"   10 seconds ago   Exited (0) 9 seconds ago             xenodochial_mclaren

$ docker image ls
REPOSITORY   TAG       IMAGE ID       CREATED       SIZE
ubuntu       latest    ca2b0f26964c   3 weeks ago   77.9MB
\end{lstlisting}

\begin{itemize}
\item Ask Docker to run a container with \texttt{ubuntu:latest}\begin{itemize}
\item I.e., latest stable version of Ubuntu Linux from \hreffoot{Docker Hub}{https://hub.docker.com/}

\end{itemize}


\item Docker can’t find a \glossref{cached} copy locally, so it downloads the image

\item Then runs it

\item But its default command is \texttt{/bin/bash} with no inputs, so it exits immediately.

\end{itemize}
\section{Re-running a Container}
\begin{lstlisting}[frame=tblr,backgroundcolor=\color{black!5}]
$ docker container run ubuntu:latest pwd
/

$ docker container run ubuntu:latest ls
bin
boot
dev
etc
home
…more entries…
sys
tmp
usr
var
\end{lstlisting}

\begin{itemize}
\item Docker doesn’t need to download the image again (it’s cached)

\item Runs the given command instead of the default \texttt{/bin/bash}

\end{itemize}
\section{This Doesn’t Work}
\begin{lstlisting}[frame=tblr,backgroundcolor=\color{black!5}]
$ docker container run ubuntu:latest "echo hello"
docker: Error response from daemon: \
failed to create task for container: \
failed to create shim task: \
OCI runtime create failed: \
runc create failed: \
unable to start container process: \
exec: "echo hello": executable file not found in $PATH: unknown.
\end{lstlisting}

\begin{itemize}
\item There is no executable in the image’s search path called \texttt{echo hello} (all one word)

\end{itemize}
\section{Pulling Images}
\begin{itemize}
\item We don’t have to run immediately

\end{itemize}
\begin{lstlisting}[frame=tblr,backgroundcolor=\color{black!5}]
$ docker pull ubuntu:latest
latest: Pulling from library/ubuntu
Digest: sha256:77906da86b60585ce12215807090eb327e7386c8fafb5402369e421f44eff17e
Status: Image is up to date for ubuntu:latest
docker.io/library/ubuntu:latest
\end{lstlisting}

\section{What Have We Got?}
\begin{lstlisting}[frame=tblr,backgroundcolor=\color{black!5}]
$ docker container run ubuntu cat /etc/os-release
PRETTY_NAME="Ubuntu 22.04.4 LTS"
NAME="Ubuntu"
VERSION_ID="22.04"
VERSION="22.04.4 LTS (Jammy Jellyfish)"
VERSION_CODENAME=jammy
ID=ubuntu
ID_LIKE=debian
HOME_URL="https://www.ubuntu.com/"
SUPPORT_URL="https://help.ubuntu.com/"
BUG_REPORT_URL="https://bugs.launchpad.net/ubuntu/"
PRIVACY_POLICY_URL="https://www.ubuntu.com/legal/terms-and-policies/privacy-policy"
UBUNTU_CODENAME=jammy
\end{lstlisting}

\begin{itemize}
\item Don’t need \texttt{:latest} every time (defaults)

\end{itemize}
\section{Inside the Container}
\begin{lstlisting}[frame=tblr,backgroundcolor=\color{black!5}]
$ docker container run -i -t ubuntu
root@4238b3b51abd:/# pwd
/
root@4238b3b51abd:/# whoami
root
root@4238b3b51abd:/# ls
bin  boot  dev  etc  home  lib  media  mnt  opt  proc  root  run  sbin  srv  sys  tmp  usr  var
root@4238b3b51abd:/# exit
exit
$
\end{lstlisting}

\begin{itemize}
\item \texttt{-i}: interactive

\item \texttt{-t}: terminal (kind of)\begin{itemize}
\item Combination often abbreviated \texttt{-it}

\end{itemize}


\item The hexadecimal number after \texttt{root@} is the container’s unique ID

\end{itemize}
\section{Persistence}
\begin{lstlisting}[frame=tblr,backgroundcolor=\color{black!5}]
$ docker container run -i -t ubuntu
root@a8ea570a84d3:/# ls /tmp
root@a8ea570a84d3:/# touch /tmp/proof-we-were-here.txt
root@a8ea570a84d3:/# ls /tmp
proof-we-were-here.txt
root@a8ea570a84d3:/# exit
exit

$ docker container run -i -t ubuntu
root@f792c15ebb5b:/# ls /tmp
root@f792c15ebb5b:/# exit
exit
\end{lstlisting}

\begin{itemize}
\item Container starts fresh each time it runs

\item Notice that the container’s ID changes each time it runs

\end{itemize}
\section{What Is Running}
\begin{lstlisting}[frame=tblr,backgroundcolor=\color{black!5}]
$ docker container ls --format "table {{.ID}}\t{{.Status}}" |
CONTAINER ID   STATUS                                       |
                                                            |
                                                            | $ docker container run -it ubuntu
                                                            | root@a5427ccdeb26:/#
                                                            |
$ docker container ls --format "table {{.ID}}\t{{.Status}}" |
CONTAINER ID   STATUS                                       |
a5427ccdeb26   Up 38 seconds                                |
                                                            |
                                                            | root@a5427ccdeb26:/# exit
                                                            | exit
                                                            |
$ docker container ls --format "table {{.ID}}\t{{.Status}}" |
CONTAINER ID   STATUS                                       |
\end{lstlisting}

\begin{itemize}
\item \texttt{docker container ls} on its own shows a wide table

\item The command uses \hreffoot{Go}{https://go.dev/} format strings for output\begin{itemize}
\item Yes, really...

\end{itemize}


\end{itemize}
\section{IDs Only}
\begin{lstlisting}[frame=tblr,backgroundcolor=\color{black!5}]
$ docker container ls -a -q
22b7c4109157
8640cfb5e07a
4c1ffdcb1c88
37f30320bc8b
fa9f02841fe9
\end{lstlisting}

\begin{itemize}
\item \texttt{-a}: all

\item \texttt{-q}: quiet

\item So \texttt{docker container rm  -f \$(docker container ls -a -q)} gets rid of everything

\end{itemize}
\section{Filtering}
\begin{lstlisting}[frame=tblr,backgroundcolor=\color{black!5}]
$ docker image ls --filter reference="ubuntu"
REPOSITORY   TAG       IMAGE ID       CREATED       SIZE
ubuntu       latest    2b7cc08dcdbb   6 weeks ago   69.2MB
\end{lstlisting}

\begin{itemize}
\item There are a \emph{lot} of Docker commands...

\end{itemize}
\section{Installing Software}
\begin{itemize}
\item Use \hreffoot{apt}{https://en.wikipedia.org/wiki/APT_(software)} (Advanced Package Tool)

\end{itemize}
\begin{lstlisting}[frame=tblr,backgroundcolor=\color{black!5}]
$ docker container run -it ubuntu

# apt update
…lots of output…

# apt install -y python3
…lots of output…

# which python

# which python3
/usr/bin/python3

# python3 --version
Python 3.10.12
\end{lstlisting}

\begin{itemize}
\item \texttt{apt update} to update available package lists

\item \texttt{apt install -y} to install the desired package\begin{itemize}
\item \texttt{-y} to answer “yes” to prompts

\item Installs lots of dependencies as well

\end{itemize}


\item Doesn’t create \texttt{python} (note lack of output)

\item Creates \texttt{python3} instead

\item Version is most recent in the default repository

\item But \emph{it isn’t there the next time we run}

\end{itemize}
\begin{lstlisting}[frame=tblr,backgroundcolor=\color{black!5}]
# exit
exit

$ docker run -it ubuntu

# which python

# exit
exit
\end{lstlisting}

\section{Actually Installing Software}
\begin{itemize}
\item Create a \glossref{Dockerfile}\begin{itemize}
\item Usually called that and in a directory of its own

\item Ours is \texttt{ubuntu-python3/Dockerfile}

\end{itemize}


\end{itemize}
\begin{lstlisting}[frame=tblr]
FROM ubuntu:latest

RUN apt update
RUN apt install python3 -y
\end{lstlisting}

\begin{itemize}
\item Tell docker to build the image

\end{itemize}
\begin{lstlisting}[frame=tblr,backgroundcolor=\color{black!5}]
$ docker build -t gvwilson/ubuntu-python3 ubuntu-python3

#0 building with "desktop-linux" instance using docker driver

#1 [internal] load build definition from Dockerfile
#1 transferring dockerfile: 99B done
#1 DONE 0.0s

#2 [internal] load metadata for docker.io/library/ubuntu:latest
#2 DONE 1.6s

#3 [internal] load .dockerignore
#3 transferring context: 2B done
#3 DONE 0.0s

#4 [1/3] FROM docker.io/library/ubuntu:latest@sha256:77906da86b60585ce12215807090eb327e7386c8fafb5402369e421f44eff17e
#4 resolve docker.io/library/ubuntu:latest@sha256:77906da86b60585ce12215807090eb327e7386c8fafb5402369e421f44eff17e done
#4 DONE 0.0s

#5 [2/3] RUN apt update
#5 CACHED

#6 [3/3] RUN apt install python3 -y
#6 CACHED

#7 exporting to image
#7 exporting layers done
#7 writing image sha256:c06d47d8275d4ef724dad192bf72daaac6b86701a1be40e1ac03f53092201d71 done
#7 naming to docker.io/gvwilson/ubuntu-python3 done
#7 DONE 0.0s
\end{lstlisting}

\begin{itemize}
\item Use \texttt{-t gvwilson/python3} to \glossref{tag} the image

\end{itemize}
\begin{lstlisting}[frame=tblr,backgroundcolor=\color{black!5}]
$ docker container run -it gvwilson/ubuntu-python3
# which python3
/usr/bin/python3
\end{lstlisting}

\section{Inspecting Containers}
\begin{lstlisting}[frame=tblr,backgroundcolor=\color{black!5}]
$ docker container inspect 56d9f83286f9
…199 lines of JSON…
\end{lstlisting}

\section{Layers}
\begin{lstlisting}[frame=tblr,backgroundcolor=\color{black!5}]
$ docker image history gvwilson/ubuntu-python3
IMAGE          CREATED        CREATED BY                                      SIZE      COMMENT
c06d47d8275d   24 hours ago   RUN /bin/sh -c apt install python3 -y # buil…   29.5MB    buildkit.dockerfile.v0
<missing>      24 hours ago   RUN /bin/sh -c apt update # buildkit            45.6MB    buildkit.dockerfile.v0
<missing>      3 weeks ago    /bin/sh -c #(nop)  CMD ["/bin/bash"]            0B
<missing>      3 weeks ago    /bin/sh -c #(nop) ADD file:07cdbabf782942af0…   69.2MB
<missing>      3 weeks ago    /bin/sh -c #(nop)  LABEL org.opencontainers.…   0B
<missing>      3 weeks ago    /bin/sh -c #(nop)  LABEL org.opencontainers.…   0B
<missing>      3 weeks ago    /bin/sh -c #(nop)  ARG LAUNCHPAD_BUILD_ARCH     0B
<missing>      3 weeks ago    /bin/sh -c #(nop)  ARG RELEASE                  0B
\end{lstlisting}

\begin{itemize}
\item Docker images are built in \glossref{layers}

\item Layers can be shared between images to reduce disk space

\end{itemize}
\begin{lstlisting}[frame=tblr,backgroundcolor=\color{black!5}]
$ docker system df
TYPE            TOTAL     ACTIVE    SIZE      RECLAIMABLE
Images          1         1         144.8MB   0B (0%)
Containers      1         0         14B       14B (100%)
Local Volumes   1         0         0B        0B
Build Cache     5         0         62B       62B
\end{lstlisting}

\begin{itemize}
\item First line (\texttt{Images}) shows actual disk space

\item The name \texttt{df} comes from a Unix command with that name to show free disk space

\end{itemize}
\section{Choosing a Command}
\begin{itemize}
\item Add \texttt{CMD} with a list of arguments to specify default command to execute when image runs

\end{itemize}
\begin{lstlisting}[frame=tblr]
FROM ubuntu:latest

RUN apt update
RUN apt install python3 -y

CMD ["python3", "--version"]
\end{lstlisting}

\begin{itemize}
\item Build

\end{itemize}
\begin{lstlisting}[frame=tblr,backgroundcolor=\color{black!5}]
$ docker build -t gvwilson/python3 python3-version
…lots of output…
\end{lstlisting}

\begin{itemize}
\item Run

\end{itemize}
\begin{lstlisting}[frame=tblr,backgroundcolor=\color{black!5}]
$ docker container run gvwilson/python3-version
Python 3.10.12
\end{lstlisting}

\begin{itemize}
\item But that’s all we get, because all we asked for was the version

\item So build a new image \texttt{gvwilson/python3-interactive} with this Dockerfile\begin{itemize}
\item Use \texttt{-i} to put Python in interactive mode

\end{itemize}


\end{itemize}
\begin{lstlisting}[frame=tblr]
FROM ubuntu:latest

RUN apt update
RUN apt install python3 -y

CMD ["python3", "-i"]
\end{lstlisting}

\begin{itemize}
\item Run it like this\begin{itemize}
\item Use \texttt{-it} to connect standard input and output to container when it runs

\end{itemize}


\end{itemize}
\begin{lstlisting}[frame=tblr,backgroundcolor=\color{black!5}]
$ docker container run -it gvwilson/python3-interactive
Python 3.10.12 (main, Nov 20 2023, 15:14:05) [GCC 11.4.0] on linux
Type "help", "copyright", "credits" or "license" for more information.
>>> print("hello")
hello
>>> exit()
$
\end{lstlisting}

\section{Copying Files Into Images}
\begin{itemize}
\item Create a new directory \texttt{python3-script} and add this file

\end{itemize}
\begin{lstlisting}[frame=tblr]
print("proof that the script was copied")
\end{lstlisting}

\begin{itemize}
\item Modify the Docker file to copy it into the image

\end{itemize}
\begin{lstlisting}[frame=tblr]
FROM ubuntu:latest

RUN apt update
RUN apt install python3 -y

COPY proof.py /home

CMD ["python3", "/home/proof.py"]
\end{lstlisting}

\begin{itemize}
\item Build and run

\end{itemize}
\begin{lstlisting}[frame=tblr,backgroundcolor=\color{black!5}]
$ docker build -t gvwilson/python3-script python3-script
…output…

$ docker container run gvwilson/python3-script
proof that the script was copied
\end{lstlisting}

\section{Order Matters}
\begin{itemize}
\item \texttt{docker build} executes Dockerfile commands in order

\item Caches each layer

\item So put things that change more frequently (like your scripts)
    \emph{after} things that change less frequently (like Linux and Python)

\end{itemize}
\section{Exericse}
\begin{enumerate}
\item 

Create a Dockerfile that installs Git
    and uses it to clone a repository containing a Python script
    as the image is being built,
    then runs that Python script by default.



\item 

What is the difference between \texttt{CMD} and \texttt{ENTRYPOINT} in Dockerfiles?
    When would you use the latter instead of the former?



\end{enumerate}
\section{Sharing Files}
\begin{itemize}
\item Containers exist to provide isolation...

\item ...but sometimes we \emph{want} interaction with external resources

\end{itemize}
\begin{lstlisting}[frame=tblr,backgroundcolor=\color{black!5}]
$ mkdir -p /tmp/mount_example

$ echo "proof that mounting works" > /tmp/mount_example/test.txt

$ docker container run -it --mount type=bind,source=/tmp/mount_example,target=/example ubuntu

# ls /example
test.txt

# cat /example/test.txt
proof that mounting works

# cp /example/test.txt /example/copied.txt

# exit

$ ls /tmp/mount_example/
copied.txt  test.txt
\end{lstlisting}

\begin{itemize}
\item To \glossref{mount} a storage device is to make its contents available
    at some location in the filesystem

\item Use \texttt{--mount} to tell Docker to make a directory of the host filesystem available
    inside the container\begin{itemize}
\item \texttt{type=bind}: there are other options (e.g., \texttt{type=volume})

\item \texttt{source=/tmp/mount\_example}: host filesystem

\item \texttt{target=/example}: where the directory appears in the container

\end{itemize}


\end{itemize}
\section{Environment Variables}
\begin{itemize}
\item Do \emph{not} put passwords in Dockerfiles \chapref{auth}

\item Common instead to pass them via environment variables\begin{itemize}
\item Which can also be used for things like server addresses

\end{itemize}


\item Build this image

\end{itemize}
\begin{lstlisting}[frame=tblr]
FROM ubuntu:latest

CMD ["echo", "variable is '${ECHO_VAR}'"]
\end{lstlisting}

\begin{itemize}
\item Run it

\end{itemize}
\begin{lstlisting}[frame=tblr,backgroundcolor=\color{black!5}]
$ ECHO_VAR=some_string docker container run gvwilson/ubuntu-env-var
variable is '${ECHO_VAR}'
\end{lstlisting}

\begin{itemize}
\item \texttt{CMD} takes the string literally

\item So try this:

\end{itemize}
\begin{lstlisting}[frame=tblr]
#!/usr/bin/env bash
echo "variable is '${ECHO_VAR}'"
\end{lstlisting}

\begin{lstlisting}[frame=tblr]
FROM ubuntu:latest

COPY show_var.sh /home
CMD ["/home/show_var.sh"]
\end{lstlisting}

\begin{lstlisting}[frame=tblr,backgroundcolor=\color{black!5}]
$ docker container run gvwilson/ubuntu-env-var-succeeds
variable is ''

$ ECHO_VAR='it worked' docker container run gvwilson/ubuntu-env-var-succeeds
variable is ''

$ ECHO_VAR='it worked' docker container run -e ECHO_VAR gvwilson/ubuntu-env-var-succeeds
variable is 'it worked'
\end{lstlisting}

\begin{itemize}
\item First time: not setting the variable

\item Second time: not telling Docker to pass that environment variable to the container

\item Third time: got it right

\end{itemize}
\section{Environment Files}
\begin{itemize}
\item Often define environment variables in a file and tell Docker to use that\begin{itemize}
\item Which means you now have to figure out how to manage a file full of secrets...

\end{itemize}


\end{itemize}
\begin{lstlisting}[frame=tblr]
ECHO_VAR=this is set in a .env file
\end{lstlisting}

\begin{lstlisting}[frame=tblr,backgroundcolor=\color{black!5}]
$ docker container run --env-file ./set_echo_var.env gvwilson/ubuntu-env-var-succeeds
variable is 'this is set in a .env file'
\end{lstlisting}

\begin{itemize}
\item Note the lack of quotes around the variable definition in the \texttt{.env} file

\end{itemize}
\section{Long-Running Containers}
\begin{lstlisting}[frame=tblr,backgroundcolor=\color{black!5}]
docker container ls -a --format "table {{.ID}}\t{{.Status}}" | head -n 5
CONTAINER ID   STATUS
56d9f83286f9   Exited (0) 3 minutes ago
3a48286cb202   Exited (0) 7 minutes ago
24e164d06d47   Exited (0) 5 hours ago
6e44181432fd   Exited (0) 5 hours ago
\end{lstlisting}

\begin{itemize}
\item Container only runs as long as its starting process runs

\item But container itself sticks around until removed

\end{itemize}
\section{Long-Lived Service}
\begin{itemize}
\item Print a count and the time every second\begin{itemize}
\item The \texttt{expr} command is rather useful

\end{itemize}


\end{itemize}
\begin{lstlisting}[frame=tblr]
#!/usr/bin/env bash
COUNTER=1
while true
do
    echo $COUNTER $(date "+%H:%M:%S")
    COUNTER=$(expr $COUNTER + 1)
    sleep 1
done
\end{lstlisting}

\begin{itemize}
\item Create a Dockerfile

\end{itemize}
\begin{lstlisting}[frame=tblr]
FROM ubuntu:latest

COPY count_time.sh /home
CMD ["/home/count_time.sh"]
\end{lstlisting}

\begin{itemize}
\item Build and run as usual

\end{itemize}
\begin{lstlisting}[frame=tblr,backgroundcolor=\color{black!5}]
$ docker container run gvwilson/count-time
1 18:38:10
2 18:38:11
3 18:38:12
4 18:38:13
…and so on…
\end{lstlisting}

\begin{itemize}
\item Cannot stop it with Ctrl-C

\item Cannot background it with Ctrl-Z

\item Only way to stop it is \texttt{docker ps} to find ID and then \texttt{docker kill}\begin{itemize}
\item Note: only have to give the first few digits of ID to \texttt{docker kill}

\end{itemize}


\end{itemize}
\begin{lstlisting}[frame=tblr,backgroundcolor=\color{black!5}]
$ docker ps
CONTAINER ID   IMAGE                 COMMAND                 CREATED         STATUS
741d896e4bb3   gvwilson/count-time   "/home/count_time.sh"   7 seconds ago   Up 6 seconds

$ docker kill 741d
\end{lstlisting}

\section{A Better Way}
\begin{lstlisting}[frame=tblr,backgroundcolor=\color{black!5}]
$ docker container run --detach gvwilson/count-time
54c8c682a94a3853c62e2f86c19d463428a01452ed7e5cf85b076dcc0f447474

$ docker ps
CONTAINER ID   IMAGE                 COMMAND                 CREATED          STATUS
54c8c682a94a   gvwilson/count-time   "/home/count_time.sh"   12 seconds ago   Up 11 seconds

$ docker container stop 54c8
…wait a few seconds…

$
\end{lstlisting}

\begin{itemize}
\item Use \texttt{--detach} to detach the container from the terminal that launched it

\item Use \texttt{docker container stop} to shut things down gracefully

\item Inspect output after the fact (or while the container is running) with \texttt{docker logs}

\end{itemize}
\begin{lstlisting}[frame=tblr,backgroundcolor=\color{black!5}]
$ docker logs 54c8
1 18:42:44
2 18:42:45
3 18:42:46
4 18:42:47
5 18:42:48
…more output…
\end{lstlisting}

\chapter{Authentication}\label{auth}




Terms defined: 
\glossref{authentication}, \glossref{authorization}, \glossref{base64 encoding}, \glossref{cleartext}, \glossref{root directory}, \glossref{root (user account)}, \glossref{superuser}


\begin{itemize}
\item See \hreffoot{Issue 16}{https://github.com/gvwilson/sys-tutorial/issues/16}

\end{itemize}
\section{What’s the Magic Word?}
\begin{itemize}
\item Only allow access to data if client:\begin{itemize}
\item \glossref{Authenticates} (i.e., establishes their identity)

\item Is \glossref{authorized} (i.e., has the right to view the data)

\end{itemize}


\item Simplest possible is wrong in many ways: does the client know a password?

\end{itemize}
\begin{lstlisting}[frame=tblr]
from http import HTTPStatus
import requests
import sys

URL = "http://localhost:8000?species=stejay"
headers = {"Password": sys.argv[1]} if len(sys.argv) == 2 else {}
response = requests.get(URL, headers=headers)
if response.status_code == HTTPStatus.OK:
    print(f"{len(response.json())} records returned")
else:
    j = response.json()
    print(f"{j['status']}: {j['error_message']}")
\end{lstlisting}

\begin{lstlisting}[frame=tblr]
python src/bird_client_password.py mumble
\end{lstlisting}

\begin{lstlisting}[frame=tblr,backgroundcolor=\color{black!5}]
16 records returned
\end{lstlisting}

\begin{lstlisting}[frame=tblr]
python src/bird_client_password.py bad_password
\end{lstlisting}

\begin{lstlisting}[frame=tblr,backgroundcolor=\color{black!5}]
403: incorrect or missing password
\end{lstlisting}

\begin{itemize}
\item First change to server: get the password on the command line and save it

\end{itemize}
\begin{lstlisting}[frame=tblr]
def main():
    sandbox, filename, password = sys.argv[1], sys.argv[2], sys.argv[3]
    os.chdir(sandbox)
    df = pd.read_csv(filename)
    serverAddress = ("", 8000)
    server = BirdServer(df, password, serverAddress, RequestHandler)
    server.serve_forever()
\end{lstlisting}

\begin{lstlisting}[frame=tblr]
class BirdServer(HTTPServer):
    def __init__(self, data, password, server_address, request_handler):
        super().__init__(server_address, request_handler)
        self._data = data
        self._password = password
\end{lstlisting}

\begin{itemize}
\item Second change: add authorization to \texttt{do\_GET}\begin{itemize}
\item Once again use our own exception class to handle unhappy cases

\end{itemize}


\end{itemize}
\begin{lstlisting}[frame=tblr]
    def do_GET(self):
        try:
            self.authorize()
            result = self.filter_data()
            as_json = result.to_json(orient="records")
            self.send_content(as_json, HTTPStatus.OK)
        except ServerException as exc:
            self.send_error(exc)
\end{lstlisting}

\begin{itemize}
\item Add authorization that checks header value

\end{itemize}
\begin{lstlisting}[frame=tblr]
    def authorize(self):
        expected = self.server._password
        actual = self.headers.get("Password", None)
        if actual != expected:
            raise ServerException("incorrect or missing password", HTTPStatus.FORBIDDEN)
\end{lstlisting}

\begin{itemize}
\item Handle errors by constructing JSON

\end{itemize}
\begin{lstlisting}[frame=tblr]
    def send_error(self, exc):
        content = {"status": exc._status, "error_message": str(exc)}
        self.send_content(json.dumps(content), exc._status)
\end{lstlisting}

\begin{itemize}
\item It works but:\begin{itemize}
\item One password for everyone

\item Sent as \glossref{cleartext} over an unencrypted connection

\end{itemize}


\end{itemize}
\section{Basic Authentication}
\begin{itemize}
\item Modify \texttt{do\_GET}

\end{itemize}
\begin{lstlisting}[frame=tblr]
    def do_GET(self):
        try:
            _ = self.authorize()
            result = self.filter_data()
            as_json = result.to_json(orient="records")
            self.send_content(as_json, HTTPStatus.OK)
        except ServerException as exc:
            self.send_error(exc)
\end{lstlisting}

\begin{itemize}
\item \hreffoot{Basic HTTP authentication}{https://en.wikipedia.org/wiki/Basic_access_authentication}:\begin{itemize}
\item Header called \texttt{"Authorization"}

\item Value is \texttt{Basic \emph{data}}

\item Data is \glossref{base-64 encoded} \texttt{\emph{username}:\emph{password}}

\end{itemize}


\item Most of the code is checking that everything is OK and responding properly if it’s not

\item Test client

\end{itemize}
\begin{lstlisting}[frame=tblr]
from http import HTTPStatus
import requests
from requests.auth import HTTPBasicAuth
import sys

URL = "http://localhost:8000?species=stejay"

user, password = sys.argv[1], sys.argv[2]
response = requests.get(URL, auth=HTTPBasicAuth(user, password))
if response.status_code == HTTPStatus.OK:
    print(f"{len(response.json())} records returned")
else:
    j = response.json()
    print(f"{j['status']}: {j['error_message']}")
\end{lstlisting}

\begin{lstlisting}[frame=tblr]
python src/bird_client_basicauth.py marlyn mumble
\end{lstlisting}

\section{Changing the Root Directory}
\begin{itemize}
\item The \texttt{chroot} command\begin{itemize}
\item Changes the process’s idea of the \glossref{root directory}

\item Runs a command

\end{itemize}


\item But...

\end{itemize}
\begin{lstlisting}[frame=tblr]
mkdir -p /tmp/filesystem
chroot /tmp/filesystem echo "it worked"
\end{lstlisting}

\begin{lstlisting}[frame=tblr,backgroundcolor=\color{black!5}]
chroot: /tmp/filesystem: Operation not permitted
\end{lstlisting}

\begin{itemize}
\item Need special permission (discussed below)

\item Everything needed to run \texttt{echo} and other commands needs to be in the new filesystem

\item Only isolates the filesystem

\end{itemize}
\section{sudo}
\begin{itemize}
\item Every machine has a \glossref{superuser} account called \glossref{root}\begin{itemize}
\item Which has nothing to do with the root directory of the filesystem

\end{itemize}


\item Use \texttt{sudo} (“superuser do”) to change identity temporarily

\end{itemize}
\begin{lstlisting}[frame=tblr]
mkdir -p /tmp/filesystem
sudo chroot /tmp/filesystem echo "it worked"
\end{lstlisting}

\begin{lstlisting}[frame=tblr,backgroundcolor=\color{black!5}]
Password: ************
chroot: echo: No such file or directory
\end{lstlisting}

\begin{itemize}
\item At least it’s a different error message...

\item \texttt{sudo} is a way to give yourself permission to mess up everything on your computer

\item So another requirement for virtual environments:
    break them without breaking anything else

\end{itemize}
\chapter{The Filesystem}\label{fs}




FIXME: describe filesystem and tools

\begin{itemize}
\item See \hreffoot{Issue 19}{https://github.com/gvwilson/sys-tutorial/issues/19}

\end{itemize}
\chapter{Access Control}\label{access}




FIXME: describe access control

\begin{itemize}
\item See \hreffoot{Issue 21}{https://github.com/gvwilson/sys-tutorial/issues/21}

\end{itemize}
\chapter{Monitoring}\label{monitor}




FIXME: describe tools for inspection and monitoring

\begin{itemize}
\item See \hreffoot{Issue 18}{https://github.com/gvwilson/sys-tutorial/issues/18}

\end{itemize}
\chapter{Running Jobs}\label{jobs}




FIXME: talk about ways to do work on demand

\begin{itemize}
\item See \hreffoot{Issue 22}{https://github.com/gvwilson/sys-tutorial/issues/22}

\end{itemize}
\chapter{Finale}\label{finale}




FIXME: write conclusion


\appendix
\chapter{License}\label{license}




All of the written material is made available under the Creative
Commons - Attribution - NonCommercial 4.0 International license (CC-BY-NC-4.0),
while the software is made available under the Hippocratic License.

\section{Writing}

\emph{This is a human-readable summary of (and not a substitute for) the license.
For the full legal text of this license, please see
\hreffoot{https://creativecommons.org/licenses/by-nc/4.0/legalcode}{https://creativecommons.org/licenses/by-nc/4.0/legalcode}.}


All of this site is made available under the terms of the Creative Commons
Attribution - NonCommercial 4.0 license. You are free to:

\begin{itemize}
\item 

\textbf{Share} — copy and redistribute the material in any medium or format



\item 

\textbf{Adapt} — remix, transform, and build upon the material



\item 

The licensor cannot revoke these freedoms as long as you follow the license
    terms.



\end{itemize}

Under the following terms:

\begin{itemize}
\item 

\textbf{Attribution} — You must give appropriate credit, provide a link to the
    license, and indicate if changes were made. You may do so in any reasonable
    manner, but not in any way that suggests the licensor endorses you or your
    use.



\item 

\textbf{NonCommercial} — You may not use the material for commercial purposes.



\item 

\textbf{No additional restrictions} — You may not apply legal terms or technological
    measures that legally restrict others from doing anything the license
    permits.



\end{itemize}

\textbf{Notices:}


You do not have to comply with the license for elements of the material in the
public domain or where your use is permitted by an applicable exception or
limitation.


No warranties are given. The license may not give you all of the permissions
necessary for your intended use. For example, other rights such as publicity,
privacy, or moral rights may limit how you use the material.

\section{Software}

Licensor hereby grants permission by this license (“License”), free of charge,
to any person or entity (the “Licensee”) obtaining a copy of this software and
associated documentation files (the “Software”), to deal in the Software without
restriction, including without limitation the rights to use, copy, modify,
merge, publish, distribute, sublicense, and/or sell copies of the Software, and
to permit persons to whom the Software is furnished to do so, subject to the
following conditions:

\begin{itemize}
\item 

The above copyright notice and this License or a subsequent version published
    on the \hreffoot{Hippocratic License Website}{https://firstdonoharm.dev/} shall be
    included in all copies or substantial portions of the Software. Licensee has
    the option of following the terms and conditions either of the above
    numbered version of this License or of any subsequent version published on
    the Hippocratic License Website.



\item 

Compliance with Human Rights Laws and Human Rights Principles:

\begin{enumerate}
\item 

Human Rights Laws. The Software shall not be used by any person or
    entity for any systems, activities, or other uses that violate any
    applicable laws, regulations, or rules that protect human, civil, labor,
    privacy, political, environmental, security, economic, due process, or
    similar rights (the “Human Rights Laws”). Where the Human Rights Laws of
    more than one jurisdiction are applicable to the use of the Software,
    the Human Rights Laws that are most protective of the individuals or
    groups harmed shall apply.



\item 

Human Rights Principles. Licensee is advised to consult the articles of
    the \hreffoot{United Nations Universal Declaration of Human Rights}{https://www.un.org/en/universal-declaration-human-rights/} and the
    \hreffoot{United Nations Global Compact}{https://www.unglobalcompact.org/what-is-gc/mission/principles} that define recognized principles
    of international human rights (the “Human Rights Principles”). It is
    Licensor’s express intent that all use of the Software be consistent
    with Human Rights Principles. If Licensor receives notification or
    otherwise learns of an alleged violation of any Human Rights Principles
    relating to Licensee’s use of the Software, Licensor may in its
    discretion and without obligation (i) (a) notify Licensee of such
    allegation and (b) allow Licensee 90 days from notification under (i)(a)
    to investigate and respond to Licensor regarding the allegation and (ii)
    (a) after the earlier of 90 days from notification under (i)(a), or
    Licensee’s response under (i)(b), notify Licensee of License termination
    and (b) allow Licensee an additional 90 days from notification under
    (ii)(a) to cease use of the Software.



\item 

Indemnity. Licensee shall hold harmless and indemnify Licensor against
    all losses, damages, liabilities, deficiencies, claims, actions,
    judgments, settlements, interest, awards, penalties, fines, costs, or
    expenses of whatever kind, including Licensor’s reasonable attorneys’
    fees, arising out of or relating to Licensee’s non-compliance with this
    License or use of the Software in violation of Human Rights Laws or
    Human Rights Principles.



\end{enumerate}


\item 

Enforceability: If any portion or provision of this License is determined to
     be invalid, illegal, or unenforceable by a court of competent jurisdiction,
     then such invalidity, illegality, or unenforceability shall not affect any
     other term or provision of this License or invalidate or render
     unenforceable such term or provision in any other jurisdiction. Upon a
     determination that any term or provision is invalid, illegal, or
     unenforceable, to the extent permitted by applicable law, the court may
     modify this License to affect the original intent of the parties as closely
     as possible. The section headings are for convenience only and are not
     intended to affect the construction or interpretation of this License. Any
     rule of construction to the effect that ambiguities are to be resolved
     against the drafting party shall not apply in interpreting this
     License. The language in this License shall be interpreted as to its fair
     meaning and not strictly for or against any party.



\end{itemize}

THE SOFTWARE IS PROVIDED “AS IS”, WITHOUT WARRANTY OF ANY KIND, EXPRESS OR
IMPLIED, INCLUDING BUT NOT LIMITED TO THE WARRANTIES OF MERCHANTABILITY, FITNESS
FOR A PARTICULAR PURPOSE AND NONINFRINGEMENT. IN NO EVENT SHALL THE AUTHORS OR
COPYRIGHT HOLDERS BE LIABLE FOR ANY CLAIM, DAMAGES OR OTHER LIABILITY, WHETHER
IN AN ACTION OF CONTRACT, TORT OR OTHERWISE, ARISING FROM, OUT OF OR IN
CONNECTION WITH THE SOFTWARE OR THE USE OR OTHER DEALINGS IN THE SOFTWARE.


\emph{The Hippocratic License is an \hreffoot{Ethical Source license}{https://ethicalsource.dev}.}

\chapter{Code of Conduct}\label{conduct}




In the interest of fostering an open and welcoming environment, we as
contributors and maintainers pledge to making participation in our project and
our community a harassment-free experience for everyone, regardless of age, body
size, disability, ethnicity, gender identity and expression, level of
experience, education, socioeconomic status, nationality, personal appearance,
race, religion, or sexual identity and orientation.

\section{Our Standards}

Examples of behavior that contributes to creating a positive environment
include:

\begin{itemize}
\item using welcoming and inclusive language,

\item being respectful of differing viewpoints and experiences,

\item gracefully accepting constructive criticism,

\item focusing on what is best for the community, and

\item showing empathy towards other community members.

\end{itemize}

Examples of unacceptable behavior by participants include:

\begin{itemize}
\item the use of sexualized language or imagery and unwelcome sexual
  attention or advances,

\item trolling, insulting/derogatory comments, and personal or political
  attacks,

\item public or private harassment,

\item publishing others’ private information, such as a physical or
  electronic address, without explicit permission, and

\item other conduct which could reasonably be considered inappropriate in
  a professional setting

\end{itemize}
\section{Our Responsibilities}

Project maintainers are responsible for clarifying the standards of acceptable
behavior and are expected to take appropriate and fair corrective action in
response to any instances of unacceptable behavior.


Project maintainers have the right and responsibility to remove, edit, or reject
comments, commits, code, wiki edits, issues, and other contributions that are
not aligned to this Code of Conduct, or to ban temporarily or permanently any
contributor for other behaviors that they deem inappropriate, threatening,
offensive, or harmful.

\section{Scope}

This Code of Conduct applies both within project spaces and in public spaces
when an individual is representing the project or its community. Examples of
representing a project or community include using an official project email
address, posting via an official social media account, or acting as an appointed
representative at an online or offline event. Representation of a project may be
further defined and clarified by project maintainers.

\section{Enforcement}

Instances of abusive, harassing, or otherwise unacceptable behavior may be
reported by emailing the project team. All complaints will be reviewed
and investigated and will result in a response that is deemed necessary and
appropriate to the circumstances. The project team is obligated to maintain
confidentiality with regard to the reporter of an incident.  Further details of
specific enforcement policies may be posted separately.


Project maintainers who do not follow or enforce the Code of Conduct in good
faith may face temporary or permanent repercussions as determined by other
members of the project’s leadership.

\section{Attribution}

This Code of Conduct is adapted from the \hreffoot{Contributor Covenant}{https://www.contributor-covenant.org/}
version 1.4.

\chapter{Contributing}\label{contrib}




Contributions are very welcome.
Please file issues or submit pull requests in our GitHub repository.
All contributors will be acknowledged.

\section{In Brief}
\begin{itemize}
\item 

Use \texttt{pip install -r requirements.txt}
    to install the packages required by the helper tools and Python examples.
    You may wish to create a new virtual environment before doing so.
    All code has been tested with Python 3.12.1.



\item 

The tutorial lives in \texttt{pages/index.md},
    which is translated into a static GitHub Pages website using \hreffoot{Ark}{https://www.dmulholl.com/docs/ark/main/}.



\item 

The source files for examples are in \texttt{src/} and the output they generate is in \texttt{out/}.



\item 

\texttt{Makefile} contains the commands used to re-run each example.
    If you add a new example,
    please add a corresponding rule in \texttt{Makefile}.



\item 

Use \texttt{[\% section\_start class="CLASS" title="TITLE" \%]}
    at the start of the first section.
    The class can be \texttt{topic} for a numbered topic,
    \texttt{aside} for an unnumbered aside,
    or \texttt{exercise} for practice exercises.
    Topics and asides must have titled;
    exercise section do not (the name is filled in automatically).



\item 

Use \texttt{[\% section\_end \%]}
    at the end of the final section.



\item 

Use \texttt{[\% section\_break class="CLASS" title="TITLE" \%]}
    to end the previous section and start a new one.



\item 

Use \texttt{[\% single "dir/file.ext" \%]}
    in \texttt{index.md} to include an arbitrary text file.
    By default, file inclusion strips out everything between \texttt{-- [keep]} and \texttt{-- [/keep]}
    for SQL files and \texttt{\# [keep]} and \texttt{\# [/keep]} for Python files.
    The start and end tags can be customized by passing \texttt{keep="label"}
    to the \texttt{single} inclusion tags.



\item 

Use \texttt{[\% multi "dir\_1/file\_1.ext" "dir\_2/file\_2.ext" ... \%]}
    to include multiple files.



\item 

Use \texttt{[\% exercise \%]} to introduce a numbered exercise.
    Do not leave a blank link between the inclusion and the text of the exercise.



\item 

Use \texttt{[\% figure file="path" title="text" alt="text" \%]} to include a numbered figure.



\item 

Use \texttt{[\% g key "text" \%]} to link to glossary entries.
    The text is inserted and highlighted;
    the key must identify an entry in \texttt{info/glossary.yml},
    which is in \hreffoot{Glosario}{https://glosario.carpentries.org/} format.



\item 

SVG diagrams are in \texttt{res/img/} and can be edited using \hreffoot{draw.io}{https://www.drawio.com/}.
    Please use 14-point Helvetica for text,
    solid 1-point black lines,
    and unfilled objects.



\item 

All external links are written using \texttt{[box][notation]} inline
    and defined in \texttt{info/tutorial.yml}.
    The shortcode \texttt{[\% link\_table \%]} at the end of \texttt{index.md} fills in these links.



\end{itemize}
\section{Logical Structure}
\begin{itemize}
\item 

Introduction

\begin{itemize}
\item A \emph{learner persona} that characterizes the intended audience in concrete terms.

\item \emph{Prerequisites} (which should be interpreted with reference to the learner persona).

\item \emph{Learning objectives} that define the tutorial’s scope.

\item \emph{Setup instructions} that instructors and learners must go through in order to code along

\end{itemize}


\item 

\emph{Topics} are numbered.
    Each contains one code sample, its output, and notes for the instructor.
    Learners are \emph{not} expected to be able to understand topics without instructor elaboration.



\item 

\emph{Asides} are not numbered,
    and contain code-less explanatory material,
    additional setup instructions,
    \emph{concept maps} summarizing recently-introduced ideas,
    etc.



\item 

\emph{Exercises} are numbered.
    An exercise section may include any number of exercises.



\item 

Topics of both kinds may contain \emph{glossary references}
    and/or \emph{explanatory diagrams}.



\item 

Appendices

\begin{itemize}
\item A \emph{glossary} that defines terms called out in the topics.

\item \emph{Acknowledgments} that point at inspirations and thank contributors.

\end{itemize}


\end{itemize}
\section{Physical Structure}
\begin{itemize}
\item \texttt{CODE\_OF\_CONDUCT.md}: source for Code of Conduct\begin{itemize}
\item \texttt{pages/conduct.md}: auxiliary file to translate CoC into HTML

\end{itemize}


\item \texttt{CONTRIBUTING.md}: this guide\begin{itemize}
\item \texttt{pages/contributing.md}: auxiliary file to translate this guide into HTML

\end{itemize}


\item \texttt{LICENSE.md}: licenses for code and prose\begin{itemize}
\item \texttt{pages/license.md}: auxiliary file to translate licenses into HTML

\end{itemize}


\item \texttt{Makefile}: commands for rebuilding examples\begin{itemize}
\item Run \texttt{make} with no arguments to see available targets

\end{itemize}


\item \texttt{README.md}: home page

\item \texttt{config.py}: Ark configuration file

\item \texttt{info/}: auxiliary data files used to build website\begin{itemize}
\item \texttt{info/glossary.yml}: glossary terms

\item \texttt{info/links.yml}: link definitions

\item \texttt{info/thanks.yml}: names of people to include in acknowledgments

\end{itemize}


\item \texttt{bin/}: helper programs (e.g., for generating databases)

\item \texttt{docs/}: generated website

\item \texttt{lib/}: Ark theme directory\begin{itemize}
\item \texttt{lib/tut/}: tutorial theme\begin{itemize}
\item \texttt{lib/tut/extensions/}: custom shortcodes

\item \texttt{lib/tut/resources/}: static files

\item \texttt{lib/tut/templates/}: the main \texttt{node.ibis} template and included files

\end{itemize}


\end{itemize}


\item \texttt{misc/}: miscellaneous files

\item \texttt{out/}: generated output files for examples

\item \texttt{requirements.txt}: \texttt{pip} requirements file to build Python environment

\item \texttt{res/}: static resources\begin{itemize}
\item \texttt{res/img/}: SVG diagrams

\end{itemize}


\item \texttt{src/}: source files for examples

\end{itemize}
\section{Tags for Issues and Pull Requests}
\begin{itemize}
\item \texttt{contribute-addition}: a pull request that contains new material

\item \texttt{contribute-change}: a pull request that changes or fixes existing material

\item \texttt{discuss}: discussion of proposed change or fix

\item \texttt{governance}: meta-discussion of project direction, etc.

\item \texttt{help-wanted}: requires knowledge or skills the core maintainer lacks

\item \texttt{in-content}: issue or PR is related to lesson content

\item \texttt{in-infrastructure}: issue or PR is related to build tools, styling, etc.

\item \texttt{report-bug}: issue reporting an error

\item \texttt{request-addition}: issue asking for new content

\item \texttt{request-change}: issue asking for a change to existing content

\end{itemize}
\section{FAQ}
\begin{description}

\item[Why computer security?] 
Because if you dig down far enough,
almost every data science project needs to get data or supply it to someone else,
and ought to know how to do that safely.

\item[Why Ark?] 
The first version of this tutorial used \hreffoot{Jekyll}{https://jekyllrb.com/}
because it is the default for \hreffoot{GitHub Pages}{https://pages.github.com/}
and because its frustrating limitations would discourage contributors
from messing around with the template instead of writing content.
However,
those limitations proved more frustrating than anticipated:
in particular,
very few data scientists speak Ruby,
so previewing changes locally required them to install and use
yet another language framework.

\item[Why Make?] 
It runs everywhere,
no other build tool is a clear successor,
and,
like Jekyll,
it’s uncomfortable enough to use that people won’t be tempted to fiddle with it
when they could be writing.

\item[Why hand-drawn figures rather than \hreffoot{Graphviz}{https://graphviz.org/} or \hreffoot{Mermaid}{https://mermaid.js.org/}?] 
Because it’s faster to Just Effing Draw than it is
to try to tweak layout parameters for text-to-diagram systems.
If you really want to make developers’ lives better,
build a diff-and-merge tool for SVG:
programmers shouldn’t have to use punchard-compatible data formats in the 21st Century
just to get the benefits of version control.

\item[Why make this tutorial freely available?] 
Because if we all give a little, we all get a lot.

\end{description}

\chapter{Bibliography}\label{bib}



\begin{description}

\item[Stoneman2020] 
Elton Stoneman.
\emph{Docker in a Month of Lunches}.
Manning, 2020.
ISBN 978-1617297052.

\end{description}

\chapter{Glossary}\label{glossary}



\begin{description}

\item[ASCII character encoding] A standard way to represent the characters commonly used in the Western European languages as 7-bit integers, now largely superceded by Unicode.

\item[authentication] The act of establishing one’s identity.

\item[authorization] The act of establishing that one has a right to access certain information.

\item[background a process] To disconnect a process from the terminal but keep it running.

\item[base64 encoding] A representation of binary data that represents each group of 6 bits as one of 64 printable characters.

\item[buffer (noun)] An area of memory used to hold data temporarily.

\item[buffer (verb)] To store something in memory temporarily, e.g., while waiting for there to be enough data to make an I/O operation worthwhile.

\item[cache] To store a copy of data locally in order to speed up access, or the data being stored.

\item[callback function] A function A that is passed to another function B so that B can call it at some later point.

\item[character encoding] A way to represent characters as bytes. Common examples include ASCII and UTF-8.

\item[child process] A process created by another process, which is called its parent process.

\item[cleartext] Text that has not been encrypted.

\item[client] A program such as a browser that sends requests to a server and does something with the response.

\item[concurrency] The ability of different parts of a system to take action at the same time.

\item[daemon] A long-lived process managed by an operating system that provides a service such as printer management to other processes.

\item[Docker] A tool for creating and managing isolated computing environments.

\item[Docker container] A particular running (or runnable) instance of a Docker image.

\item[Docker image] A package containing the software and supporting files Docker needs to run an application in isolation.

\item[Dockerfile] The name usually given to a file containing commands to build a Docker image.

\item[dynamic content] Web site content that is generated on the fly. Dynamic content is usually customized according to the requester’s identity, query parameter, etc.

\item[encryption] The process of converting data from a representation that anyone can read to one that can only be read by someone with the right algorithm and/or key.

\item[environment variable] A shell variable that is inherited by child processes

\item[filesystem] The set of files and directories making up a computer’s permanent storage, or the software component used to manage them.

\item[flush] To move data from a buffer to its intended destination immediately.

\item[foreground a process] To reconnect a process to the terminal after it has been backgrounded or suspended.

\item[fork (a process)] To create a duplicate of an existing process, typically in order to run a new program.

\item[header (of HTTP request or response)] A name-value pair at the start of an HTTP request or response. Headers are used to specify what data formats the sender can handle, the date and time the message was sent, and so on.

\item[hostname] A human-readable name for a computer on a network.

\item[HTTP] The protocol used to exchange information between browsers and websites, and more generally between other clients and servers. Communication consists of requests and responses.

\item[HTTP method] The verb in an HTTP request that defines what the client wants to do. Common methods are \texttt{GET} (to get data) and \texttt{POST} (to submit data).

\item[HTTP request] A precisely-formatted block of text sent from a client such as a browser to a server that specifies what resource is being requested, what data formats the client will accept, etc.

\item[HTTP response] A precisely-formatted block of text sent from a server back to a client in reply to a request.

\item[HTTP status code] A numerical code that indicates what happened when an HTTP request was processed, such as 200 (OK), 404 (not found), or 500 (internal server error).

\item[JavaScript Object Notation (JSON)] A way to represent data by combining basic values like numbers and character strings in lists and key-value structures. Unlike other formats, it is unencumbered by a syntax for writing comments.

\item[layer (of Docker image)] FIXME

\item[local server] A server running on the programmer’s own computer, typically for development purposes.

\item[localhost] A special host name that identifies the computer that the software is running on.

\item[MIME type] A standard that defines types of file content, such as \texttt{text/plain} for plain text and \texttt{image/jpeg} for JPEG images.

\item[mount] FIXME

\item[name collision] The problem that occurs when two different applications use the same name for different things.

\item[operating system (OS)] A program whose job is to manage the hardware of a computer. Other programs interact with the OS through system calls.

\item[parent process] A process which has created one or more other processes, which are called its child processes.

\item[path (in filesystem)] An expression that refers to a file or directory in a filesystem.

\item[port] A logical endpoint for communication, like a phone number in an office building.

\item[process] A running instance of a program.

\item[process ID] The unique numerical identifier of a running process.

\item[process tree] The set of processes created directly or indirectly by one process and the parent-child relationships between them.

\item[query parameter] A key-value pair included in a URL that the server may use to modify or customize its response.

\item[refactor] To reorganize code without changing its overall behavior.

\item[resolve (a path)] To translate a path into the canonical name of the file or directory it refers to.

\item[resume (a process)] To continue the execution of a suspended process.

\item[root (user account)] The usual ID of the superuser account on a computer.

\item[root directory] The top-most directory in the filesystem that contains all other directories and files.

\item[sandbox] An isolated computing environment in which operations can be executed safely.

\item[server] A program that waits for requests from clients and sends them data in response.

\item[shell] A program that allows a user to interact with a computer’s operating system and other programs through a textual user interface.

\item[shell variable] A variable set and used in the shell.

\item[shell\_script] A program that uses shell commands as its programming language.

\item[signal] A message sent to a running process separate from its normal execution, such as an interrupt or a timer notification.

\item[signal handler] A callback function that is called when a process receives a signal.

\item[source (in shell script)] To run one shell script in the same process as another.

\item[static file] Web site content that is stored as a file on disk that is served as-is. Serving static files is usually faster than generating dynamic content, but can only be done if what’s wanted is unchanging and known in advance.

\item[superuser] An administrative account on a computer that has permission to see, change, and run everything.

\item[suspend (a process)] To pause the execution of a process but leave it intact so that it can resume later.

\item[system call] A call to one of the functions provided by an operating system.

\item[tag (a Docker image)] FIXME

\item[Unicode] A standard that defines numeric codes for many thousands of characters and symbols. Unicode does not define how those numbers are stored; that is done by standards like UTF-8.

\item[UTF-8] A way to store the numeric codes representing Unicode characters that is backward-compatible with the older ASCII standard.

\item[virtual environment] A set of libraries, applications, and other resources that are isolated from the main system and other virtual environments.

\item[web scraping] The act of extracting data from HTML pages on the web.

\end{description}

\chapter{About the Author}\label{author}




\textbf{\hreffoot{Greg Wilson}{https://third-bit.com/}} has worked in industry and academia for over 40 years
and is the author, co-author, or editor of several books,
including \emph{Beautiful Code},
\emph{The Architecture of Open Source Applications},
\emph{JavaScript for Data Science},
\emph{Teaching Tech Together},
and \emph{Research Software Engineering with Python}.
He was the co-founder and first Executive Director of Software Carpentry
and received ACM SIGSOFT’s Influential Educator Award in 2020.

\chapter{Colophon}\label{colophon}



\begin{itemize}
\item 

The tutorial text uses \hreffoot{Atkinson Hyperlegible}{https://brailleinstitute.org/freefont},
    which was designed to be easy for people with impaired vision to read.
    Code uses Source Code Pro and diagrams use Helvetica.



\item 

The colors in this theme
    are lightened versions of those used in \hreffoot{classic Canadian postage stamps}{https://third-bit.com/colophon/}.
    The art in the title is by \hreffoot{Danielle Navarro}{https://art.djnavarro.net/}
    and used with her gracious permission.



\item 

The CSS files used to style code were obtained from \hreffoot{highlight-css}{https://numist.github.io/highlight-css/};
    legibility was checked using \hreffoot{WebAIM WAVE}{https://wave.webaim.org/}.



\item 

Diagrams were created with the desktop version of \hreffoot{draw.io}{https://www.drawio.com/}.



\item 

The site is hosted on \hreffoot{GitHub Pages}{https://pages.github.com/}.



\item 

Traffic statistics are collected using \hreffoot{Plausible}{https://plausible.io/},
    which provides a lightweight ethical alternative to surveillance capitalism.



\item 

Thanks to the authors of  \hreffoot{Ark}{https://www.dmulholl.com/docs/ark/main/},
    \hreffoot{BeautifulSoup}{https://pypi.org/project/beautifulsoup4/},
    \hreffoot{html5validator}{https://pypi.org/project/html5validator/},
    \hreffoot{pybtex}{https://pypi.org/project/pybtex/},
    \hreffoot{ruff}{https://pypi.org/project/ruff/},
    and all the other software used in this project.
    If we all give a little,
    we all get a lot.



\end{itemize}
\printindex
\end{document}

